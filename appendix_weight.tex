\section{Weight clustering \label{app:appendixWeight}}
A better way to quantify the seekers and parasites in the social system is to do 
a clustering analysis based on the two dimensional space of the weight.
Each agent can be represented as a point in a two dimensional space where the x-axis
represents the weight of the food seeking behaviour and the y-axis represents the weight
of the  parasitic seeking behaviour.
Initially because all the agents have the same predictive weight, there will be $N$ points
in the same location, during the specialization process there will be a cloud of points.
The cloud will have a different shape according to the proportion of seekers and parasites.
For instance seekers will be described as points spread on an horizontal cluster, whereas 
parasites will be spread on a vertical cluster.
The larger the variance, the better the specialization, therefore PCA \nomenclature{PCA}{Principal Component Analysis} 
can be used to quantify the degree of specialization by looking at the main component of the space.
For example in Fig.\ref{fig:weightCluster} I have plotted the evolution of the points during learning for two different cases: 
\begin{itemize}
 \item $N=20$ agents and $M=4$ food sources: at the end of the simulation there are 16 parasites identified by the horizontal cluster 
 \item $N=20$ agents and $M=18$ food sources: at the end of the simulation there are 16 seekers identified by the vertical cluster 
\end{itemize}
When the system differentiates in half seekers and half parasites, there is only one cluster
with a circular shape.  

\begin{figure}[htbp]
\begin{center}
\includegraphics[scale=0.3]{figures/WeightCluster.eps}
\end{center}
\vspace*{4pt}
\caption[Cluster analysis of the weights]{Weight evolution in a two-dimension space
for two important cases: on the first row there are 20 agents for only 4 food sources,
on the second row there are 20 agents for 18 food sources.
Each plot contains on the x-axis the synaptic weight for the seeking behaviour and on the
y-axis the synaptic weight for the food seeking behaviour.

\label{fig:weightCluster}}
\end{figure}
