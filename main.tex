\documentclass[a4paper,10pt]{article}

\usepackage{amssymb}
\usepackage{amsmath}
\usepackage[dvips]{graphicx}
   \graphicspath{{figures/}}
  % and their extensions so you won't have to specify these with
  % every instance of \includegraphics
   \DeclareGraphicsExtensions{.eps}
%opening
\title{Some notes}
\author{Paolo Di Prodi}

\begin{document}

\maketitle

\begin{abstract}

\end{abstract}







\section{A new model for our agent}
To be able to apply some more advanced modeling of our system at the individual and system level we had to change our simulation.
The agent is now modeled as a semi-discrete adaptive controller:
\begin{itemize}
 \item orientation is discretized in 8 directions: $\theta_{0,...,7}={0,45,90,135,180,225,270,315}$ which corresponds to the cardinal sectors $E,NE,N,NW,W,SW,S,SE$
 \item sensor range is discretized in 8 sectors where the first sector is centered in $\alpha_0=0$ with an aperture of 45 degree.
 \item the input for every sector is the minima of the inverse distance with another object
 \item the agent has  constant translational speed $v$
 \item the predictive controller (ICO) outpus the angular velocity of the agent
\end{itemize}
The world where the agents are is still continuous area where the origin is placed in top left corner.
The input for every sector is low pass filtered as:
\begin{equation}
 formula here
\end{equation}

The reflex error for the agent $ith$ can be discretized as a random variable $X_i$, the predictive error for the same agent is discretized as a random variable $Y_i$.
\begin{figure}[!htbp]
\includegraphics[scale=0.4]{agent}
\caption{}
\end{figure} 
In Gibson's ecological approach [arg missing this reference!]





%\include{appendix}

\bibliography{infotheory,robot}
\bibliographystyle{plain}
\end{document}
