
This Thesis describes the work I have done on the theory of social systems
which produced the following publications:
\begin{itemize}
\item Di Prodi, P. and Porr, B. and W\"{o}rg\"{o}tter, F. (2010) A Novel Information
Measure for Predictive Learning in a Social System Setting. From Animals to
Animats 11, Lecture Notes in Computer Science, editors: Doncieux, Stéphane and
Girard, Benoît and Guillot, Agnès and Hallam, John and Meyer, Jean-Arcady and
Mouret, Jean-Baptiste, Springer Berlin / Heidelberg, 511-522, vol, 6226.

\item Di Prodi P, Porr B and W\"{o}rg\"{o}tter F (2009). A novel information measure to
understand differentiation in social systems. Frontiers in Computational
Neuroscience. Conference Abstract: Bernstein Conference on Computational
Neuroscience. doi: 10.3389/conf.neuro.10.2009.14.040

\item Kulvicius T, Kolodziejski C, Prodi P, Tamosiunaite M, Porr B and
W\"{o}rg\"{o}tter F (2008). On the analysis and evaluation of closed loop learning
systems. Frontiers in Computational Neuroscience. Conference Abstract: Bernstein
Symposium 2008. doi: 10.3389/conf.neuro.10.2008.01.078

\item Di Prodi, P. Porr, B. W\"{o}rg\"{o}tter, F. (2008) Adaptive Communication
Promotes Sub-system Formation in a Multi Agent System with Limited Resources.
In: Learning and Adaptive Behaviors for Robotic Systems, 2008. LAB-RS '08. ECSIS
Symposium
\end{itemize}

Additionally, I have done some research work in collaboration, with previous PhD
student Lynsey McCabe, in the field of computational neuroscience which produced
the following publications:
\begin{itemize}
\item Bernd Porr, Lynsey McCabe, Paolo Di Prodi, Christoph Kolodziejski,
Florentin W\"{o}rg\"{o}tter, How feedback inhibition shapes spike-timing-dependent
plasticity and its implications for recent Schizophrenia models, Neural
Networks, In Press, Corrected Proof, Available online 10 March 2011, ISSN
0893-6080, DOI: 10.1016/j.neunet.2011.03.004.

\item McCabe L., Porr B., Di Prodi P. \& W\"{o}rg\"{o}tter F. (2008) OBSERVING STDP
OF PYRAMIDAL CELL AND ATTACHED INTERNEURON MICROCIRCUIT USING DETAILED CA2+
DYNAMICS, Fens Forum 2008

\item McCabe, L., Di Prodi, P., Porr, B. and W\"{o}rg\"{o}tter, F. (2007) Shaping of
STDP curve by interneuron and Ca2+ dynamics. Proceedings of the sixteenth annual
computational neuroscience meeting CNS*2007, Toronto.
\end{itemize}

I have also done some work on the application of machine user interfaces for
health services which produced the following publications however they are not relevant
with my PhD:
\begin{itemize}
\item Di Prodi P, Power CF and Wei PY (2009). Extending the reach of Mental
Health Services through eLearning technology and other communication mediums
centralized on one Online ePlatform. Frontiers in Neuroengineering. Conference
Abstract: Annual CyberTherapy and CyberPsychology 2009 conference. doi:
10.3389/conf.neuro.14.2009.06.030

\item Power CF, Di Prodi P. Extending the reach of Mental Health Services
through eLearning technology and other communication mediums centralized on one
Online ePlatform. ISBE 2008

\end{itemize}
