
\section{Simmetry breaking in social tasks}

The approach can be also applied to my model, with a few variations, and I
 can describe the procedure that could be used in the future to verify
 the symmetry breaking property.
By keeping the same notation N agents are faced with a binary decision between
 being a S (seeker) or a P (parasite).
The symmetry parameter is defined as $s(t)=L(t)/N$, where L is the number of
agents which became seekers at time t.
A majority decision is any steady state of the system where at least $L \geq \delta N$
agents have became seekers with $0.5 \ll \delta \leq 1.0$.
That means that $s \geq \delta$ and $s$ is the only parameter required
because $s+p=N$ and thus $p(t)$ is not required.
If $s(t)$ converges as demonstrated in the previous experiments, it is always
possible to find the steady state probability density function (PDF) for $s(t)$
noted as $p^*(s)$.
The PDF $p^*(s)$ estimates how many agents will become seekers.
By integrating the PDF function we can estimate the proportion P of
experiments in which a majority decision with at least $\delta$ majority occurs:
\begin{equation}
\int_0^\delta p^*(s) \mathrm{d}s + \int_{1-\delta}^1 p^*(s) \mathrm{d}s=P
\end{equation}
To find $p^*(s)$ is necessary to run a large number of parallel simulations
with different initial conditions and then to run a statistical analysis.
To avoid this computational cost, the author \citet{Hamann2010:modelsimmetry} makes an assumption
about the nature of $s(t)$ by assuming that is described by a
mono dimensional Langevin equation, which is a form of stochastic differential
equation:
\begin{equation}
\frac{ds}{dt}=\alpha(s,t)+\beta(s,t) \xi(t)
\label{eq:langevin}
\end{equation}
where:
\begin{itemize}
 \item $\alpha$ is the deterministic development or drift
 \item $\xi$ is the Gaussian noise $|\xi(t)|=1$, with mean $<\xi(t)>=0$,
and uncorrelated in time $<\xi(t)\xi(t')>=\delta(t-t')$
 \item $\beta$ is the fluctuation of the the noise amplitude
\end{itemize}
It is not possible to assume that such a mono dimensional
description exists for every high dimensional system, thus
the approach assumes that such a description exists
and then an estimation of $\alpha,\beta$ is feasible with some heuristic formula
or with some numerical fitting strategies.
Once the two parameters are estimated we need to accept or
reject the hypothesis by comparing the statistical property of
$s(t)$ generated by the detailed simulations with the solution
$s(t)$ of the Langevin equation with the estimated parameters.

The most common heuristic to estimate the parameters was
used in \citet{Hamann2010:AnalysisSimmetry} and is based on two features of $s(t)$.
The first is the mean of the absolute changes:
\begin{equation}
\varDelta s^{abs}(s,t)=\frac{1}{K}\sum_i | s_i(t)-s_i(t-1)|
\end{equation}
averaged over K samples $s_i(t)$ from many independent
simulations runs.
The second is the mean of the relative changes:
\begin{equation}
\varDelta s^{rel}(s,t)=\frac{1}{K}\sum_i  s_i(t)-s_i(t-1)
\end{equation}
which is an approximation of the derivative because
it contains the difference between two time steps.
Unfortunately the author did not have any time left for running
additional simulations and computing the mean and absolute
and relative change so he can only speculate about what their outcome would be.
The $\varDelta s^{rel}(s,t)$ gives an indication of the
stability of the system by computing how many zero crossings
the function has.
Each zero crossing indicates that the derivative is zero
and thus a steady state was achieved for that particular
configuration.
The heuristic to estimate $\alpha,\beta$ is built on the discretised
version of the Langevin equation \ref{eq:langevin}:
\begin{equation}
s_{t+1}=s_{t}+ \varDelta s^{rel}(s_t)+( \varDelta s^{abs}(s_t)-|\varDelta s^{rel}(s_t)|)\xi_t
\end{equation}
where $\xi_t$ is again the Gaussian white noise.
The white noise is a general approximation but one could calculate the second
moments of $\varDelta$ like variance for each time step.
The Fokker–Planck equation can be used for computing the probability density
for a stochastic process described by a stochastic differential equation:
\begin{equation}
\frac{\partial \rho_s}{\partial t}=\frac{\partial}{\partial s}(\alpha(s,t)\rho_s)+\frac{1}{2} \frac{\partial^2}{\partial s^2}(\beta^2(s,t)\rho_s)
\end{equation}
to obtain the time development of the probability density function for s and
thus its steady state PDF.
The parameters  (drift and diffusion coefficient) are thus:
\begin{align*}
\alpha(s,t)= \varDelta s^{rel}(s_t)
\beta(s,t)=\varDelta s^{abs}(s_t)-|\varDelta s^{rel}(s_t)|
\end{align*}
To validate the model, one needs to compute the PDF from the simulations
and from the solution of the Fokker-Plank equation.
If the model is valid we should see something like in Figure \ref{fig:conclusion:simmetryexpected}:

\begin{figure}[htbp]
\begin{center}
%\includegraphics[scale=0.3]{}[MISSING]
\end{center}
\vspace*{4pt}
\caption[Expected result of simmetry]{
This is an example of the expected stochastic solution \label{fig:conclusion:simmetryexpected}}
\end{figure}

The Figure was generated intuitively and shows also one of the
trajectories that were computed during previous simulations.
If the model is not valid, there are several possible explanations:
\begin{itemize}
 \item the heuristic to compute $\alpha,\beta$ was not valid
 \item the system cannot be described by a mono dimensional Langevin equation
\end{itemize}
The Fokker-Planck equation -if valid- can then be used to quantify
the effectiveness of the symmetry breaking by computing the steady state $p^*(s)=0$
and the stability of the decision by computing the splitting probabilities $\pi_w(x)$.

In summary, the statistic modelling approach is a powerful tool for the analysis
of self-organising systems.
However it cannot be applied to systems where there are more then
two decisions. 