\documentclass[a4paper,10pt]{article}
\usepackage[utf8x]{inputenc}

%opening
\title{}
\author{Paolo}

\begin{document}

\maketitle

\begin{abstract}

\end{abstract}

\section{Social order generation by double contingency}
Dittrich [reference] modelled and simulated the situation of double contingency as the origin of social
order.
He starts from an dyadic social interaction and then expands it to a multi agent interaction.
Social order appears in the dyadic social interaction and also in the multi agent but in only certain conditions
that we are going to explain.

The double contingency problem esists when a dyad composed of 2 entities (actors) meet each other:
each actor has a double nature of knower and object of cognition.

Parsons solved the problem of contingency by imposing that a common shared symbol system
is a pre-condition for the formation of a social order.
Therefore the dyad must share a culture derived from an history of previous relationships.

Lumann solves the problem of double contingency by using a self-organization processs which
develops in time based on mutual expectations.
The only assumptions he made was that alter and ego (the actors in the dyad) have a necessity of ``expectation-certainty'', that means Alter and Ego want to know what is going on in this interaction.
Every entity expects that the other entity has expectations about its next activity.

The desire -if we can call it- of every agent during the interaction is to reduce the entropy -uncertainty- of the
Alter's actions given Ego's actions.

To make a simple example when we say ``Hello'' to a friend we expect to have an ``Hello`` followed by 
a handshake and a possible conversation. The friend will expect the same.
This is useful as we don't have to try out all our vocabulary every time to get somebody's attention.

So the model proposed by Dittrich begins with a simple dyad of 2 actors:
\begin{itemize}
 \item Ego initializes the interaction by choosing a message from a set of N possible onse
 \item Alter receives the message and reply with another one from the set of N
 \item the conversation continues from Ego and so on
\end{itemize}
The activity of an actor is to decide which message has to be sent given a received message. 
This kind of action is a reduced model of Luhmann's communication because there is no distinction between
information, trasmission and understanding.
Ego chooses the message and sends it (information+ transmission) and reply to a message without understanding it.
Also agents do not show meaningful motived behaviour towards otehrs according to goals and a shared symbolic systems as proposed in the Parson's model.

Each agent is motivated by 2 functions:
\begin{itemize}
 \item Expectation-Expectation EE: an agent wants to meet the expectations of the other agent. It does so by keeping a memory of what actions were chosen in response to other agents.
 \item Expectation-Certainty EC:  the reaction of the other agent following its own activity should be as predictable as possible.
 \item Activity Value: a linear combination of $(1 - \alpha)EE$ + $\alpha EC$.
 \item Activity Probability: a parametrized version of the activity value $\gamma$ that can go from deterministic $\gamma \rightarrow \infty$ to probabilistic $\gamma \rightarrow 0$ .
\end{itemize}
Each agent chooses the activity which maximizes the activity probability according to the parameter $\gamma$.
The parameter $\alpha$ accounts in a way the ''selfishness`` of the agent because when $\alpha=1$ the agent will choose the action that will produce the most likely reaction from Alter whereas when $\alpha=0$ the agent will choose the action that Alter will expect more.

The problem is then to measure Social Order and see if the dyadic condition is able to produce high values of social order. There are 2 different point of views: a system view and an individual view.

At the individual level we can re-use the EC function to compute how certain an agent is when it selects a message. 
The average certainty $O_AV$ has a high value when certainty is high and thus indicates high social order.

At the system level we can measure:
\begin{itemize}
 \item the average number of different activities $N_D$ selected during the time interval: the lower the number the higher the order. An observer will deduce that high social order is achieved if agents always select the same activities out of a vast selection set.
 \item predictability of an activity $O_p$ or social integration: it measures how predictable an activity of a randomly drawn agent Ego is, given the activity presented on the sign by another randomly drawn agent Alter.
\end{itemize}

We can interpret the $O_p$ value as an index of the pattern formation in the behaviour of action selection. The actors formed a closed system of interaction because they are mutually predictive, this closure indicates a first order separation degree between a system and its environment.

For the dyadic case social order as measured either by $N_D$ and $O_AV$ emerges for any parameter setting in $\alpha,\gamma, N$ with stable activity patterns followed robust to small disturbances.
Measuring social integration for the dyadic case is trivial because there are only 2 agents which interacted with each other and thus it will be $O_p=1$.

The big challenge is then to scale up the dyadic case to the multi agent case.
If we have a population of few $M$ agents and we choose a random selection strategy to pair interactions, 
social order is high in terms of $O_P$ because is possible to predict very well each agent's reaction.
However at the individual level $O_AV$ is low because each agent is using the same memory to predict interactions with different agents. As a consequence increasing $M$ decreases the $O_P$ system order.

Dittritch discovered that there are 2 changes required to produce high social order with increasing number of agents:
\begin{itemize}
 \item agents must calculate Expectation-Expectation from observation of the interaction of other agents
 \item agents must use only Expectation-Expectation for activity selection ($\alpha=0$)
\end{itemize}

At the individual level agents are cognitive entities able to perceive, memorize, generalize and to make predictions.
For society to emerge they must be able to observe the interactions between others. 

\section{Trust and social order}

\section{Expected oriented modelling}
\end{document}
