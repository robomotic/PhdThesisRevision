\section{Introduction to the theory of Societies \label{Chapter1:SocialSystemTheory}}

This section contains an introduction to the theories about the generation of
human like societies and existing models in literature which replicate
artificial societies.

\section{A brief history of sociology}
Sociology is the study of society and aims to understand how social order is
possible.
In the last 350 years, sociologists have provided different explanations to the
generation of social order:
\begin{itemize}
 \item \citet{Thomas1651:Leviathan} attributed the generation to
a powerful state, the Leviathan
 \item \citet{Smith1776:WealthNations} attributed the generation
to an ``invisible hand''
 \item \citet{Durkheim1893} introduced the concept of norms
 \item \citet{Parsons1937:StructureSocialAim} extended the
Durkheim proposal by saying that norms are legitimated by values located in a
cultural system of a society
 \item \citet{Axelrod1984} suggested that the generation is
possible by rational choice of action with consideration for a long common future
(shadow)
\end{itemize}
A cardinal point in the theory of social order generation was introduced by
Parson as the \textit{the problem of double contingency}:
\begin{quote}
There are two crucial reference points for analysing interactions:
(1) That each actor is both acting agent and object of orientation both to
himself and to the others; and (2) that, as an acting agent orients to himself and
to others, in all primary modes of aspect. The actor is knower and object
of cognition, utiliser of instrumental means and himself a means, emotionally
attached to others and an object of attachment, evaluator and object of
evaluation, interpreter of symbols and himself a symbol. 
(International Encyclopedia of the Social Sciences,1968: 436)
\end{quote}

Following \citet{Talcott1968:SocioTheory,Luhmann95}
identified \textit{the problem of double contingency} as the main problem of
producing social order whilst expanding the idea with a new radical biological
principle called autopoiesis.

\subsection{Autopoiesis: from biology to social systems}
Luhmann based his social system theory on the concept of ``autopoiesis``,
originally formulated in biology by the two biologists \citet{Maturana1980}.
The biological concept of autopoiesis (from the Greek autos=self, poiein= to produce)
states that a living system recursively reproduces its elements through its own elements.
A living cell, for example, reproduces its own elements, like proteins or
and a plant grows its own leaves and roots.
The autopoietic system has 3 important properties and is described in Figure \ref{Fig:Autopoiesis:System}):
\begin{enumerate}
 \item operative closure: no operations can enter nor leave the system within its boundary
 \item interactional openness: the system has contact with its environment by means of disturbances
 \item structural coupling: environmental events can trigger internal processes but the internal
processes triggered are determined by the structures of the system
\end{enumerate}
The operative closure is symbolised by the partial feedback of the output to the 
input of the cell.
In essence autopoietic systems are at the same time open and closed systems;
open because they are influenced by their environment,
but also closed because environment does not directly
influence the structure and elementary processes of the systems.

The first and second property are also the basis of cognition:
\begin{quote}
Living systems are cognitive systems, and living as process is a process of cognition.
(Maturana and Varela 1980:13)
\end{quote}
The operations of an autopoietic system are defined as its cognitions: cognition is
a self-referential, autopoietic process.
This assumption is known as Radical Constructivism:
all ideas are constructs of the cognitive system and are a by-product of reality.
The most important contribution to the development of constructivism to neuro biological
 systems was pioneered by \citet{VanFoerster2003:Cybernetics}.
Every organism lives in a closed loop with its own environment and works only with neural activity.
The Figure \ref{Fig:Autopoiesis:Radical} shows how the cognitive system produces neural activity and
perceives neural activity: the environment is constructed as part of the feedback loop.
This self-reference property also exists at other layers, for instance in a call 
proteins create other proteins but the environment is coded as the protein production 
to maintain a membrane (made again by proteins).
This assumption is also important in the work of this thesis because artificial
agents, as well as human beings, construct their own reality. For instance a common behaviour
between animals is the natural reflex reaction to pain: when the skin is touching a
hot surface, the nervous system produces the idea of pain or heat.
The molecular property of the flame, triggered an action potential in the pain
receptor of the skin that then sent a reaction command in the motor cortex.
The physical event did not enter the cognitive system but only generated
a disturbance from the plateau state of the nervous system.

\begin{figure}[htbp]
\begin{center}
\includegraphics[width=0.8 \textwidth]{autopoiesis/neuralfeedback}
\end{center}
\small{
\caption[Radical constructivism example]{Radical constructivism example:
a burning flame is a physical system,
regulated by an oxygen reaction, when a finger touches it, a coupling is established
between the neural system and the flame.
The pain receptor produces neural activity, the brain generates a motor reaction, which then
 evokes an absence of sensory pain indicating that the reaction was good. 
The sensor produces again a different neural activity to encode the absence of pain.
\label{Fig:Autopoiesis:Radical}}}
\end{figure}

Later, in Section \ref{Section:ICOlearning} there is a clear explanation of the
implication of such an assumption.

The third property is the concept of self-organisation: the autopoietic system replicates
its elements by following a structure which is self-determined.
Thus we can  state that autopoiesis refers to the reproduction of the elements
and self-organisation refers to the determination of structures \citep{Luhmann2000}.

\begin{figure}[htbp]
\begin{center}
\includegraphics[width=0.8 \textwidth]{autopoiesis/maturanabiology}
\end{center}
\small{
\caption[Biological definition of autopoiesis]{The original definition given by Maturana and Varela:
An autopoietic machine is a machine organized (defined as a unity) as a network
of processes of production (transformation and destruction) of components which:
(a) through their interactions and transformations continuously regenerate and
realize the network of processes (relations) that produced them; and (b) constitute
 it (the machine) as a concrete unity in space in which they (the components)
exist by specifying the topological domain of its realization as such a network.
$[\dots]$ the space defined by an autopoietic system is self-contained and cannot be
described by using dimensions that define another space. When we refer to our
interactions with a concrete autopoietic system, however, we project this system
 on the space of our manipulations and make a description of this projection.
From \citet{Maturana1980} reproduced with permission of the publisher.
\label{Fig:Autopoiesis:System}}}
\end{figure}

Luhmann generalises the principle of autopoiesis to be a general form of system building
thus declaring that a system is autopoietic when it reproduces its own elements.
Figure \ref{Fig:Autopoiesis:General} describes the new hierarchy of autopoietic systems:
\begin{enumerate}
 \item level 1 contains the general definition
 \item level 2 contains living systems, psychic systems, neural systems and social systems
 \item level 3 contains societies, organisations and interactions
\end{enumerate}
Each system is described by the unit self-reproducible elements, in particular neural systems 
reproduces their own neural activity, living systems their own proteins and societies
 their own communications. 
In this thesis we are going to focus on the study of Societies \citep{Luhmann95}
which reproduce
themselves on the basis of communication and on the study of Neural Systems.
Moreover, Luhmann has also investigated the formation of organisations \citep{Luhmann2000} 
and social interactions \citep{Luhmann1993}.

\begin{figure}[htbp]
\begin{center}
\includegraphics[width=0.8 \textwidth]{autopoiesis/general}
\end{center}
\small{
\caption[General definition of autopoiesis]{Autopoiesis becomes a general
concept applied at different layers of
system formation. This thesis is focusing on studying the Social Systems.
\label{Fig:Autopoiesis:General}}}
\end{figure}

The next sections contains a descriptions of the neural systems that constitutes
the core of our artificial agents or robots.


\subsection{Neural systems \label{Introduction:NeuralSystem}}

The fundamental component of almost every living organism is its nervous system which
 is necessary for the the most vital activities like digestion, reproduction to the 
most complex ones like locomotion and cognition.
The most important assumption was made by \citet{VonFoerster85}, one of the founders of radical constructivism,
 who argued that the nervous systems is operationally closed: neural activity generates 
other neural activity and thus the environment is only a ''simulacra``.
The most basic type of a neural system is the reactive system, in control theory 
known also as a feedback system, which only reacts after a sensory event has occurred.
A feedback system can also be called a closed loop or self-referential system,
but I am going to use the most popular notation of closed loop system from now on.

\begin{figure}[htbp]
\begin{center}
\includegraphics[width=0.6 \textwidth]{neural/closedloop-1}
\end{center}
\small{
\caption[Closed loop reactive system]{
A simple closed loop system. 
Top: The transfer function $H_0$ transforms sensor signals
$X_0$ into motor signals $V$. The transfer function $P_0$ transforms the motor signals $V$
back into sensor signals $X_0$.
Bottom: the system can be simplified by dividing everything by $P_0$
\label{Fig:Neural:ControlReactive}}}
\end{figure}

The block diagram in Figure \ref{Fig:Neural:ControlReactive} represents a minimal system with feedback.
Each block contains the Laplace transform of the time domain function transfer:
\begin{itemize}
 \item $H_0(s)$ is the transfer function of the agent
 \item $P_0(s)$ is the transfer function of the environment
\end{itemize}
The transfer functions operates on neural signals which are the basic elements
 of the neurological system.
Because the systems is in closed loop form the following equations are valid:
\begin{eqnarray*}
  V(s)= H_0(s) \cdot X_0(s)\\
  X_0(s)=P_0(s) \cdot V(s) 
\end{eqnarray*}
Such a closed loop system can be stable or unstable and \citep{VonFoerster85} assumes,
like in traditional control theory, that it must operate in the stable state by 
using a negative feedback.
Stability is necessary in a linear system if the organism wants to reach a desired
 state after a finite time.
The main issue with this closed loop model is that the organism cannot distinguish
 itself from the environment because by dividing both the transfer functions by $P_0$,
the new transfer function $G(s)=H_0/P_0$ (see Figure \ref{Fig:Neural:ControlReactive}, bottom) 
does not distinguish any more from organism and environment due to the unity gain feedback. 
To avoid this issue, a disturbance $D$ must be defined as in Figure \ref{Fig:Neural:ReactiveDisturbance}.
The disturbance is anything which prevents the organism to keep its stable desired state 
and was also formulated by \citep{Ashby1956:IntroCybernetics} in its law of requisite variety which is going to
 be discussed in Section \ref{Chapter6:Information Flow}.

\begin{figure}[htbp]
\begin{center}
\includegraphics[width=0.8 \textwidth]{neural/closedloop-2}
\end{center}
\small{
\caption[Closed loop reactive system with disturbance]{
The closed loop system now contains a disturbance from the environment.
An example is the classical natural reaction to a burning sensation:
 the hand retracts when the pain receptor is activated by the contact
with the flame. Figure used with permisison from \citet{Porr2006cf}.
\label{Fig:Neural:ReactiveDisturbance}}}
\end{figure}

The disturbance this time cannot be eliminated by dividing the transfer functions by $P_0$.
A simple reactive system then can only react after a change it the desired state, 
a simple example is the motor reaction of our hand to a painful event like touching a 
flame. 
Another important point made by the radical constructivism theory is that only actions which 
feed back to the organism's sensors can be observed by the organism. 
In the mentioned example then even a more complex motor reaction will only generate 
two possible state at the sensory input, one of pain and one of non pain.
Any other action which simply disappears in the environment cannot be
observed by the organism. Thus, there is no other chance for the organism as to
analyse its inputs as this is the only aspect that the organism is able to observe.
Even its own actions are only observable through its inputs.

A more advanced organism is one which has an anticipatory input signal $X_1$ as
 shown in Figure \ref{Fig:Neural:ProactiveDisturbance}:
when the agent is born only the inner loop $H_0,P_0$ is active but by using the 
association between $X_0$ and $X_1$ the agent can in the future avoid the 
painful signal by only using the anticipatory information $X_1$.
The reflex signal $X_0$ represents the initial behavioural
goal \citep{Verschure98summary} where $X_0=0$, while the predictive signal $X_1$
is provided naturally by the environment or the organism's sensor setup.

\begin{figure}[htbp]
\begin{center}
\includegraphics[width=0.8 \textwidth]{neural/closedloop-3}
\end{center}
\small{
\caption[Closed loop proactive system with disturbance]{
The inner feedback loop is established by the transfer functions $H_0$
and $P_0$. 
The outer feedback loop is established by the transfer functions $H_1$, $P_{01}$ 
and $P_1$. $D$ is the disturbance and $T$ delays the disturbance. 
The outer feedback loop observes the inner feedback loop and adjusts
so that the inner reflex loop is no longer needed. Figure used with permisison from \citet{Porr2006cf}.
\label{Fig:Neural:ProactiveDisturbance}}}
\end{figure}

The desired state, for example $X_0=0$ cannot be maintained all the time
as disturbances $D$ arrive at the loop occasionally. These disturbances enter
the inner loop delayed by time $T$. The undelayed disturbances enter the organism
via the sensor input $X_1$ which anticipates the input $X_0$ The signal at $X_1$
can now be used to observe the primary feedback loop ($X_0$,$P_0$) and determine 
what is the effect of on the primary feedback loop. This observation can be used to adjust
$H_1$ in a way that the inner feedback loop does not feel the disturbance any more.
This approach was succesfully implemented by \citet{PorrNecoISO2003} where a 
robot was using a camera system to learn the avoidance of obstacles.
The camera provided the visual anticipatory information required to predict the 
triggering of the touch avoidance sensor.
The robot used a learning statregy called $ISO$ learning which uses the correlation between
 vision and touch signals to learn the anticipatory reaction.
A more advanced form of $ISO$ learning called $ICO$ learning is used in this thesis as the main controller
 both for the software agent and the embodied robot.

The most interesting implication of this model when using multiple agents is the 
double contingency whereby each agent is disturbing the other as shown in Figure 
\ref{Fig:Neural:MutualDisturbance}.
Each agent is nested in multiple closed loops, because it receives the outputs of 
all the other agents in the environment via the disturbance summation with the 
environment itself.
Double contigency happens when all agents are learning continuously with, for example
ICO \citep{Porr2006ICO} or ISO \citep{PorrNecoISO2003} learning , from each other.
This learning loop was defined by Luhmann as the problem of double contigency:
an open ended interaction process where each agent try to predict the other.
A good metaphor is the game of chess where each player try to anticipate its opponent 
moves in the future: in a way the chess game is a sort of communicative process between
the two players which eventually comes to an end when the desired state (check mate) 
is achieved.
This dynamic process constitutes for Luhmann the necessary base for social system generation.
It is possible that learning will never stabilise due to such multiple nested
 closed loops.

\begin{figure}[htbp]
\begin{center}
\includegraphics[width=0.8 \textwidth]{neural/closedloop-4}
\end{center}
\small{
\caption[Double contigency]{
Each organism output is summed to the motor feedback loop with the 
disturbance from the environment.
\label{Fig:Neural:MutualDisturbance}}}
\end{figure} 

\subsection{Breitenberg vehicles \label{Intro:Braitenberg}}

A good controller is useless if it cannot be embodied in a physical robot.
The ICO and ISO controller were successfully implemented on a moving robot by
 adopting the approach developed by \citet{Braitenberg1986} who devised a simple
 yet effective method to implement avoidance and attraction behaviours on 
simple robots.
The most basic architecture is composed by a couple of sensory inputs, a couple of motors
and a matrix of connectivity which defines the behaviour of the robot.
The test case scenario is a vehicle moving on a planar surface which contains a light source
 be projected from above.
The vehicle has 2 light sensor which are able to measure the scalar gradient of 
the light source on the surface.
The motors are directly connected to the sensors via positive or negative proportional 
controllers.
Figure \ref{Fig:Braitenberg:Example1} shows that an avoidance behaviour can be implemented
 by using positive feedback connections:

\begin{itemize}
 \item a vehicle with symmetric connections from sensors to motors, dislikes the source 
of light by constantly avoiding it.
 \item a vehicle with crossed connections from sensors to motors, dislikes the source of light
but constantly turns toward it at maximum speed, as if it wanted to destroy it.
\end{itemize}

\begin{figure}[htbp]
\begin{center}
\includegraphics[width=0.8 \textwidth]{vehicles/Braitenberg-1}
\end{center}
\small{
\caption[Braitenberg vehicles positive feedback]{
An example of a coward and aggressive vehicle with only positive feedback connections.
\label{Fig:Braitenberg:Example1}}}
\end{figure}

Figure \ref{Fig:Braitenberg:Example2} shows that an attraction behaviour can be implemented
 by using negative feedback connections:
\begin{itemize}
 \item a vehicle with symmetric connections from sensors to motors, likes the source 
of light by carefully approaching it and reaching ideally zero velocity in proximity.
 \item a vehicle with crossed connections from sensors to motors, likes the source of light
but constantly search for other alternatives 
\end{itemize}

\begin{figure}[htbp]
\begin{center}
\includegraphics[width=0.8 \textwidth]{vehicles/Braitenberg-2}
\end{center}
\small{
\caption[Braitenberg vehicles negative feedback]{
An example of a lover and explorer vehicle with only negative feedback connections.
\label{Fig:Braitenberg:Example2}}}
\end{figure}

The behaviour of such vehicles can be enriched by including more sensors and so
 mixing positive and negative connections or by using non monotonic relationships
 between sensors and motors.
The agents that I used in the software simulations and robot implementations, 
are based on the Braitenberg framework but with a minor difference in the implementation:
instead of having a direct connection between inputs and motors, an error signal
is generated between the left and right synapse which then generates the differential
motor command for the left and right wheel viat the ICO learning algorithm.
 
\subsection{Communication}
For Luhmann, the most important feature of a social system is that communication
is the basic form of the autopoietic reproduction of social systems.
Social systems consist of communicative processes (see Figure
\ref{Fig:Autopoiesis:Communication}), not human beings.

\begin{figure}[htbp]
\begin{center}
\includegraphics[width=0.8 \textwidth]{autopoiesis/elements}
\end{center}
\small{
\caption[Communication is the basis for society]{Living systems are based
on cellular mechanisms of reproduction and
their elements are proteins and molecules. Social systems are based on communication
processes which reproduces themselves. Psychic systems are based on self-reproducing
thoughts. Each system is structurally coupled with each other.
The structural coupling between psychic systems and communications is established
through Language. Language allows the ''synchronization`` between the two systems
but communication is also possible without it.
\label{Fig:Autopoiesis:Communication}}}
\end{figure}

Human beings are individuals in a society only when they communicate and
thus the limit of society is defined by the limit of communication.
Communication has been defined by Luhmann as a unity of 3 selective
and independent processes:
\begin{itemize}
 \item utterance: how do we utter it?
 \item understanding: how do we separate utterance from information?
 \item information: what was the utterance about?
\end{itemize}
Communication is an \textbf{emergent property} of the interaction between two psychic systems
as the three selections do not belong to each individual independently.
Each step is considered by Luhmann as a selection process from a set of possibilities,
in accordance with the definition of information of Shannon and Weaver \citep{Shannon1948}.
To explain the communication process Figure \ref{Fig:Autopoiesis:CommExample}
contains an example of a communication session between Ego and Alter.
The psychic system embodied in Ego has the intention or desired state of being alone
 because he/she is tired.
The first step for Ego is to select the appropriate utterance which in this case 
is the English sentence ''go away``.
At this point Alter receives the utterance and needs to understand it by generating the 
information from it.
Alter then can decide to leave or to continue the conversation. Assuming that in the first
case, he leaves Ego alone and emits an utterance like ''Oky``, Ego will understand that 
its initial utterance ''go away`` was successful.
On the contrary if Alter persist and does not leave Ego alone, Ego will try
 indefinitely to produce other utterances until Alter leaves.
 
In Figure \ref{Fig:Autopoiesis:CommExample} Ego can also accept or reject the
meaning of the communication which implies a dynamic selection mechanism for
the continuation or interruption of the communication process.

\begin{figure}[htbp]
\begin{center}
\includegraphics[width=0.8 \textwidth]{autopoiesis/dialogexample}
\end{center}
\small{
\caption[Communication between alter and ego]{
Ego is a psychic system with a current goal or desired state.
When Ego is in contact with Alter in a contingent situation. he selects the utterance 
''go away`` to be communicated to Alter. Alter then needs to extract the 
meaning from such an utterance and infers that Ego wants to be alone. 
Then Alter can decide to continue in the conversation or quit it by just leaving
 Ego alone.
The acceptance or rejection of the understanding is an option in the communication.
\label{Fig:Autopoiesis:CommExample}}}
\end{figure}

The Luhmann interpretation of communication is therefore far more advanced
than the simple channel communication theory where the purpose is to transfer
meaning with minimal errors.
In this sense there is similarity with the work of \citep{Berger1975:URT} 
who developed an axiomatic theory to explain people's attempts to “make sense” of interpersonal
situations by reducing uncertainty through seeking information.
Additionally the concept of reducing uncertainty
during the understanding process being a motivation to continue a communication 
is not mentioned in the original work.
There is also another interesting implication of using Luhmann's communication 
which fits recent theories of the mind.
The use of expectations is tightly linked with the idea of estimating each other's
state of mind as described with greater detail in Section \ref{TheoryOfMind} and \ref{Conclusion:Language}.
So in summary, communications can reproduce themselves but what communications
are produced is related to the concept of expectation.
In the next section I am going to introduce the most relevant simulation model 
which captures the social ordered generation by using such a model of communication 
based on expectations.

\section{Social order generation by double contingency \label{Introduction:SocialOrderModel}}

The starting point for implementing a simulation which uses Luhmann's communication approach, 
was developed by \citet{SocialOrderScalability} which implemented the situation of
double contingency as the origin of social order.
The results produced by this approach were very promising


\subsection{Methods}
He started from a dyadic social interaction and then expanded it to a multi agent interaction.
Social order appears in the dyadic social interaction and also in the multi agent situation,
but only in certain conditions that we are going to explain.

The double contingency problem exists when a dyad composed of 2 entities (ego and alter) meet each other:
each actor has a double role of knower and object of cognition.

Parsons solved the problem of contingency by asserting that a common shared symbol system
is a pre-condition for the formation of a social order.
Therefore the dyad must share a culture derived from a history of previous relationships.

Luhmann solves the problem of double contingency by using a self-organisation process which
develops in time based on mutual expectations.
The only hypothesis he made was that alter and ego (the actors in the dyad) have a
necessity of predicting ``expectation-certainty'', which means Alter and Ego want to know what
is going on in this interaction.
Every entity expects that the other entity has expectations about its next activity.

The desire -or goal- of every agent during the interaction is to reduce the entropy -uncertainty- of the
Alter's actions given Ego's actions.

To give a simple example, when I say ``Hello'' to a friend, I expect to receive a ``Hello`` followed by
a form of embrace, typically a handshake possibly followed by a conversation. My friend will expect the same.
This is useful as we don't have to try out all our vocabulary every time to get somebody's attention!
Figure \ref{Fig:Dittrich:Model} explains the model with a simple example where
Ego says ''Good morning`` and alter replies with ''Good morning``, then Ego
asks ''How are you?'' but Alter ask a question about time.
This somehow puzzles Ego -delusion- because he wasn't expecting a question after
his question.
\begin{figure}[htbp]
\begin{center}
\includegraphics[width=0.8 \textwidth]{autopoiesis/expectations}
\end{center}
\small{
\caption[Expectations and communication structures]{
Upper block: an example of mutual expectation in a conversation.
Lower block: the simulation model with the Expectation-Expectation memory
and the Expectation-Certainty model.
\label{Fig:Dittrich:Model}}}
\end{figure}

So the model proposed by Dittrich begins with a simple dyad of 2 actors:
\begin{itemize}
 \item Ego initializes the interaction by choosing a message from a set of N possible ones
 \item Alter receives the message and replies with another one from the set of N
 \item the conversation continues from Ego and so on
\end{itemize}
The activity of an actor is to decide which message has to be sent given a received message.
Each agent is motivated by 2 functions:
\begin{itemize}
 \item Expectation-Expectation \textbf{EE}: an agent wants to meet the
expectations of the other agent. It does so by keeping a memory of what actions
were chosen in response to other agents.
 \item Expectation-Certainty \textbf{EC}:  the reaction of the other agent
following its own activity should be as predictable as possible.
 \item Activity Value: a linear combination of $(1 - \alpha)EE + \alpha EC$.
 \item Activity Probability: a parameterised version of the activity
value $\gamma$ that can go from deterministic $\gamma \rightarrow \infty$
to probabilistic $\gamma \rightarrow 0$ .
\end{itemize}

Each agent chooses the activity which maximises the activity probability according
 to the parameter $\gamma$.
The parameter $\alpha$ accounts in a way the ''selfishness`` of the agent because
when $\alpha=1$ the agent will choose the action that will produce the most likely
reaction from Alter whereas when $\alpha=0$ the agent will choose the action
that Alter will expect more.

The problem is then to measure Social Order and see if the dyadic condition is
able to produce high values of social order. There are 2 different points
of view: a system view and an individual view.

At the individual level we can re-use the EC function to compute how certain
an agent is when it selects a message.
The average certainty $O_AV$ has a high value when certainty is high and thus
indicates high social order.

At the system level we can measure:
\begin{itemize}
 \item the average number of different activities $N_D$ selected during the
time interval: the lower the number the higher the order. An observer will
deduce that high social order is achieved if agents always select the same
activities out of a vast selection set.
 \item predictability of an activity $O_p$ or social integration: it measures
how predictable an activity of a randomly drawn agent Ego is, given the
activity presented on the sign by another randomly drawn agent Alter.
\end{itemize}

We can interpret the $O_p$ value as an index of the pattern formation in
the behaviour of action selection. The actors formed a closed system of
interaction because they are required to develop mutually predictive trust,
this closure indicates a first order separation degree between a system and its environment.

\subsection{Results}
For the dyadic case social order as measured either by $N_D$ and $O_AV$ emerges
for any parameter setting in $\alpha,\gamma, N$ with stable activity patterns
following robust to small disturbances.
Measuring social integration for the dyadic case is trivial because there are
only 2 agents who interacted with each other and thus it will be $O_p=1$.

The big challenge is then to scale up the dyadic case to the multi agent case.
If we have a population of few $M$ agents and we choose a random selection strategy
to pair interactions,
social order is high in terms of $O_P$ because is possible to predict each agent's 
reaction with a high degree of accuracy.
However at the individual level $O_AV$ is low because each agent is using the
 same memory to predict interactions with different agents.
As a consequence increasing $M$ decreases the $O_P$ system order.

Dittritch discovered that there are 2 changes required to produce high social
order with an increasing number of agents:
\begin{itemize}
 \item agents must calculate Expectation-Expectation from observation of the interaction of other agents
 \item agents must use only Expectation-Expectation for activity selection ($\alpha=0$)
\end{itemize}

At the individual level agents are cognitive entities able to perceive, memorise,
generalise and to make predictions.
For society to emerge they must be able to observe the interactions between others.

\subsection{Discussion}
The results of such a model were quite promising but there are also some weaknesses:
\begin{itemize}
 \item the system operates exclusively on a symbolic communicative system
 \item the system does not distinguish between actions which manipulate the environment 
and communications 
 \item the system is discrete and cannot operate in the analog domain
\end{itemize}

The mentioned issues were the main drives for the model described in this thesis because 
if such a social system needs to be implemented in the real world, the agents 
or robots needs to have a separate layer, one for the actions in the environment and one
 for the communicative events which are essentially symbolics.
Another desired property is the use of analogic based controllers which can 
then be implemented in very fast reactive electronic controllers or also 
digitized and implemented on a micro controller.
In the next section, I am going to describe the framework used to achieve such targets.



