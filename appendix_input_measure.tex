
\section{Input correlation method}
\subsection{Cross correlation corrections}
\label{Appendix:crosscorr}
Cross correlation measure cannot be applied directly to variable time windows,
thus correction terms based on the number of samples $N$ must be included.
Corrected cross correlation measures are: biased, unbiased, normalised.
Biased estimate of the cross correlation function is:
\begin{equation}
 R_{xy,biased}(m)=\frac{1}{N} R_{xy}(m)
\end{equation}
Unbiased estimate of the cross-correlation function is:
\begin{equation}
 R_{xy,unbiased}(m)=\frac{1}{N-|m|} R_{xy}(m)
\end{equation}
Normalised cross correlation is:
\begin{equation}
c_{xy}(m)=\frac{1}{N-m} \sum_{1}^{N-m} \frac{x_{i}-\bar{x}}{\sigma_{x}} \frac{y_{i}-\bar{y}}{\sigma_{y}}
\end{equation}
where $\bar(x)$ and $\sigma_{x}$ denotes mean and variance of $x$, and $m$ is a time lag.
\subsection{Coherence function}
In Laplace domain cross correlation is defined as:
\begin{equation}
 C_{xy}(\omega)=(Fx)(\omega)(Fy)*(\omega)
\end{equation}
where $(Fx)$ is the Fourier  transform of x, $\omega$ are the discrete
frequencies ($-N/2<\omega<N/2$) and $*$ means complex conjugation.
The cross spectrum $C_{xy}(w)$ is a complex number whose normalised amplitude:
\begin{equation}
 L_{xy}(\omega)=\frac{|<C_{xy(w)}>|}{\sqrt{<C_{xx}(w)>}\sqrt{<C_{yy}(w)>}}
\end{equation}
is called the coherence function and gives a measure of the linear synchronization
 between x and y as a function of the frequency $\omega$. This measure
is very useful when synchronisation is limited to some particular frequency
band, as it's usually the case of EEG signals (see \citep{EEGxcorr} for a review]).
\subsection{Energy and power of digital signal}
\label{app:energy}
For the digital signal $c_{d}(m,k)$, where $d$ is the synapse direction ($d=\{left,right\}$), $m$
is the value of the cross correlation in the time window of index $k=0,1,..$ with $N_{s}$ samples:
\begin{itemize}
\item Energy: $E(c_{d}(k))=\sum_{m=1}^{N_{s}} (c_{d}(m,k))^2$
\item Power: $P(c_{d}(k))=\frac{\sum_{m=1}^{N_{s}} (c_{d}(m,k))^2}{N_{s}}$
\end{itemize}


\subsection{Alternative measures for analog signals}
One can apply for instance synchronisation measures for EEG signals
(for a review \citep{PhysRevE.65.041903}) even if we are not interested to
detect the (driver-response) relationships between signals. Indeed we already know
that $x_{1}$ precedes $x_{0}$ in the avoidance case because this relation is a
physical causal property of the environment. However these measures should
equally work, with an extra computational cost:
\begin{enumerate}
\item linear measures:
\begin{enumerate}
\item normalized cross-correlation
\item coherence function
\end{enumerate}
\item non-linear measures:
\begin{enumerate}
\item S,H,N
\item Mutual information and transfer entropy
\item Hilbert phase analysis
\item Wavelet phase analysis
\end{enumerate}
\end{enumerate}

According to these EEG studies it has been shown that cross correlation is less
 sensible than non-linear measures because EEG signals are produced by
non-linear systems, so actual simulations are investigating if this property
 is true so too in our multi agent system.

\subsection{Alternative measures for discrete time series}
If we are dealing with discrete signals like spike trains, temporal information
can be computed by:ISI distance, Rosendal distance,Euclidean distance,
cross-correlograms, joint peristimulus PSH.
