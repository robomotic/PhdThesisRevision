\section{Summary of results}
This thesis has developed a computational model for the implementation
of artificial societies based on the theoretic foundation of Luhmann.
The societies are based on software agents which learn simple avoidance
and attraction behaviours by means of a biologically inspired Hebbian rule.
The intra agent communication is based on a minimalist implementation
of Luhmann's communication model and is simple enough to generate the self-organising
behaviour of social sub-division.
The social division was assessed initially by looking at the synaptic weight development
of each agent individually.
This approach implies that the agents are considered as white or grey boxes, which
means that the approach is not feasible if I consider a general agent whose
internal status is not accessible (black box).
Therefore I have then firstly developed two input measures called \textbf{maxcorr} and \textbf{AI},
secondly input/output measures \textbf{MI} to reflect the complexity reduction
in their agent's behaviour and a single called \textbf{Predictive Performance} to gauge the learning 
performance of the agent.
The measure has proven successful in measuring not only the behavioural reduction in
a social setting but also the learning performance of the agent.
This measure is general enough to be applied also to other learning controllers,
for instance a Q-learning avoiding robot.
The strength of this approach compared to previous work is that it is
an information based measure that can be applied to real agents
as well as simulated agents.
Previous models in literature are based on discretised models or strategic games.
In the following section \ref{Conclusion:Discussion}, I am going to described the
modelling choices used in the thesis.
After the discussion there is a series of 
In section \ref{Conclusion:FutureWork} there is a description of possible extensions 
of the social system model, a potential application and a better analysis based on
model checking.
In section \ref{Conclusion:Thoughts}, I introduce some interesting topics as well as philosophical
questions about the relationship between information theory, the theory of mind,
psychology, language and neuroscience.
In section \ref{Conclusion:Industrial} there is a small overview of the existing
commercial systems relying on social robotic systems.

\subsection{Discussion}
\label{Conclusion:Discussion}
The most relevant work that was done in the past about the implementation
of Luhmann's principle in a computational model is the one of \citet{SocialOrderScalability}.
The model is implemented as a language game where agents are learning
during mutual interactions.
The social model that was developed in this thesis, although it does not include
 a complete communication protocol described by Luhmann and used by Dittrich,
it is capable of generating sub-systems.
It also has mainly 2 advantages:
\begin{itemize}
 \item it is a real time simulator where the agents interact and learn continuously
 from each other rather then being limited to a simulated game.
 \item it can be implemented on a real robotic system as described in the Appendix 
\end{itemize}

The other advantage, compared to \citet{SocialOrderScalability}, is that
the agents integrate action and communication in a very transparent manner
thus facilitating a future expansion of the system.
The communication used in the model is mono directional and one to many:
this has an operative advantage in terms of fast response times if the
system has to adapt to environmental changes.

As a comparative analysis, previous works in information measures was performed by Polani,
 Lungarella and Ay that used information theoretic cost functions to optimise the agent's behaviour:
\begin{itemize}
\item \citet{organizationInfo} evolves controllers to maximise the information transfer
of the sensory-motor loop (empowerment) and discovers that to use memory efficiently they perform 
compression as in Figure \ref{fig:conclusion:polani}.
\item \citet{LungarellaInformation} uses mutual information to generate information structures by motor feedback as in Figure \ref{fig:conclusion:lungarella}.
\item \citet{AyClosedLoop} maximises the excess entropy (the mutual information between past and present) 
 of the agent's input thereby changing the controller's parameter to achieve a 
 working regime (exploratory and sensitive to the environment) for the robot as in Figure \ref{fig:conclusion:ay}.
\end{itemize}

There is a substantial difference between the afore mentioned approaches and the
one used in this thesis:
the predictive performance discussed in Section \ref{Chapter8:Predictive Performance}
calculates the learning ability of a general adaptive controller
 based on the information flow.
This is different from the other study by \citet{organizationInfo,LungarellaInformationStructure,AyClosedLoop} 
which use the information flow as a reward signal for the agent to learn.
Nevertheless there are compatible results that shows how the
two approaches are complementary.
For example the experiment made by \citet{LungarellaInformationStructure}
where he computed entropy measures on a saliency-driven attention
task, where a camera foveates red blocks. The entropy for the
foveation case is less than the random case:
this means that a closed loop system induces statistical regularities
in the information flow.
My results coming from the social system application yields
a similar concept: agents regularise their inputs
by selecting information which affects their motor behaviour.
There is a mutual relation between perception, information and action:
agents select the information which in turns change their
behaviour and their predictability.
Indeed in Fig.\ref{social:learning1}(D) the system is unstable
when every agent is using all the information and thus producing
non predictive behaviour. But when agents start to select the
relevant information, they simplify their behaviour and intrinsically reinforce
the stability of the system since agents mutually benefit from the increased predictability.
Therefore the Predictive Performance is also used to measure the degree of behavioural
selection of each agent during the social division as also described in Section \ref{Chapter8:PPsocial}.

Another interesting experimental work in the field of performance measure was produced 
by \citet{Kulvicius2009:analysisdifferential} where the measure was the argument of the maximum 
cross correlation between the antenna's events $\bar{\tau}$ in a fixed time window.
Kulvicius shows that for more complex environments $\vec{\tau}$ has a larger
deviation compared to the case of more simple environments.

The predictive performance can also be applied to reinforcement learning \citep{TD}
as demonstrated in Section \ref{Chapter7:Q learning application} where
the information flow was computed for a simple robot avoiding obstacles.
Although the author did not have time to compute the predictive performance, 
only the information flow, there are no evident limits that will stop
the computation of the predictive performance.

The predictive performance was finally applied to the social system scenario
as demonstrated in Section \ref{Chapter8:PPsocial} to show how agents
select either behaviour in terms of information flow.
This analysis show how the agents are selecting the information path and 
how is related to their weights' development.
It also shows that performance cannot be based on the analysis of the weights
but can only be reliably assessed via the predictive performance.

The following section describes what modelling choices were done during the
research and how they were justified.

\subsection{Modelling choices}
In the theory of social communication \citet{Luhmann95} and previous cybernetic
 experts asserted that an artificial agent or organism does not have mutual expectations
 from other entities except than other agents. An example which explains this condition is
 the difference interaction between a person and an object or a person with another person.
We have a direct expectation for an object because we know that it only adheres to the
 laws of gravity and we know that throwing it will make a parabolic trajectory in the air.
We have a mutual expectation between ``ego'' me and ``alter'' you, because we both try
 to predict what we are thinking. In a way that is more similar to a chess game where each
 player tries to predict the next move of the other in order to maximize his chance of victory.
In the deterministic memory based model developed by \citet{SocialOrderScalability}, social
 order arose from the social interaction of agents.
Therefore my choice in the model was to separate the prediction of the world/environment from the
 prediction of other agent's actions.
The identification of alter and ego was not included in the model and is left for future work,
however most advanced organisms like mammals and primates are able to recognize their
interlocutor and so maintain different expectations according to their previous interactions.
The language used to communicate the food information is similar to a sign language which is based
on the action layer, the most simple example could be the everyday traffic flow of cars.
The left light arrow indicates that the car in front of us is going to turn left and vice-versa.
This kind of sign language is used in the animal world and has been extensively studied by 
\citet{AnimalSignals}.
A good example is the evolution of the ritualisation of the mating and fighting behaviour
as briefly described in Figure \ref{fig:conclusion:ritualizationmodel}.

\begin{figure}[htbp]
\begin{center}
\includegraphics[scale=0.3]{figures/conclusion/Ritualization-3.eps}
\end{center}
\vspace*{4pt}
\caption[Ritualization model]{This diagram shows how Ethologist explain
the evolution of animal signalling or language 
\label{fig:conclusion:ritualizationmodel}}
\end{figure}

A more advanced communication language uses a symbolic language which is based
on top of the action or sign language:
primates developed a more advanced mean of communication using sound and developing
 specialized areas of the brain like the ``Brocha area'' in humans and equivalent
structures in the singing birds \citep{FoxP2Gene:Nature:2001,FOXP2Identification:2005}.

A sign or ritualised language is constrained by the environment and cannot develop further, 
this is why a better model would require the use of a symbolic language which I am
going to describe in the next section.

\subsection{A theory of language}
\label{Conclusion:Language}

This section contains first an introduction to the theories regarding the development
of human and animal languages.
It contains a brief summary of the current knowledge about the biological
roots of language formation and experiments which try to replicate artificial
language.

The communication model used in the social systems is very basic and mimics essentially
the simple mechanism of signals used in simple animals like primates or event insects.
The power of a more abstract or symbolic language has been assessed in lesion brain
studies where basically it was discovered that intelligence is rooted in language.
The recent discovery of the FOXP2 gene -dubbed the "language gene"- has provided
an astonishing example of the importance of the Broca's functional area;
in humans, mutations of FOXP2 cause a severe speech and language
disorder \citep{FoxP2Gene:Nature:2001,FOXP2Identification:2005}.

A model of language development, was achieved by \citet{Steels:1998:OriginsOntologies,Steels:1999:TalkingHeads}, 
the talking heads experiment shows that a grounded language can indeed be
evolved and contain many properties seen in natural languages like polysemy and
synonyms.
\citet{Steels:1999:TalkingHeads} has investigated how artificial agents can self-organize
languages with natural-language like properties and how meaning can co-evolve
with language. His hypothesis is that language is a
complex adaptive system that emerges through adaptive interactions between
agents and continues to evolve in order to remain adapted to the needs and
capabilities of the agents.
Thus a community of language capable agents can be viewed as a complex adaptive
system which collectively solves the problem of developing a shared
communication system.
To achieve that, the community must reach an agreement on a set of forms
(a sound system in the case of spoken language), a set of meanings (the
conceptualisations of reality), and a set of form-meaning pairs (the
lexicon and grammar).
The experiments implemented interactive robots that were programmed to play language games and
observed the characteristics of the languages that emerged; surprisingly the
agents were able to self organize and develop a common language, which resembled many features of human like
languages, without the help of an external teacher. 
The pre-requisite for the emergence of language is the cognitive and
 sensory-motor ability at the individual level; without the ad-hoc apparatus for
exchanging information and the ability to categorize the environment it is impossible to develop a language.

The experiments shows that for a language to emerge there are several conditions:
\begin{itemize}
 \item a common frame of attention
 \item the ability of perspective change
 \item a reliable system of communication
 \item adaptive learning
\end{itemize}

A sign language can emerge by interacting agents in the world: organisms can develop
an alphabet of actions that will generate a predictive behaviour. Put simply, the agents have mutual 
expectancies from each other: if agent A sees a red square
in front of him, it will produce a tone say at 150 Hz and agent B maybe will produce
a tone say at 300 Hz. With time the agents will use a common alphabet to indicate
different geometric shapes. This is possible because (A) both agents have the same or
similar computational capacities and (B) both agents can imitate each others actions.
Imitation is very important for the development of language and it has been discovered
that this important function is implemented by mirror neurons both in primates and
humans \citep{Buccino200:MirrorNeurons}. Even non primates like birds, must have a brain circuit
generated by the gene PX64 to enable them to reproduce acoustic sounds of similar pitch.
Birds that doesn't have this gene cannot develop a common language, even children who
lack this gene are not able to develop a proper language. It turns out that
if an agent is to be able to learn a language they must have a capacity to imitate and process in memory 
actions performed by his peers.

My current communication model on the contrary is limited or bounded to the properties
of the environment: doing so limits the recursivity of a symbolic system.
I need a higher process that is able to abstract the embodied
actions' that happen in the world to a higher level of actions (let's call them symbols
but they can be sounds or tones) that are able not only to refer to the objects of
the real world but also to each other.

It is very easy to see the limit of an embodied language system: a well known studied
effect is that one of the foraging domestic pigeons (which google sarcastically claims
use a new “page ranking”system). Pigeons do have a good visual system but can
detect food relatively well on the ground, once a pigeon finds a possible source of food
it goes on the ground and randomly samples the ground to find the lucky spot!
Another pigeon flying nearby will see his fellow pigeon wondering around the interesting
spot and with some probability it will go on the ground and look for the food itself!
If we repeat the step many times for each time a pigeon passes nearby we can see an
entire storm of pigeons eating “imaginary” food! This kind of behaviour [reference]
is a positive feedback mechanism which is based on a priory knowledge of the pigeon:
the fact that if a fellow pigeon is looking for food on the ground then there should
be something there! To some extent pigeons base their decisions on Bayesian inference.
Surprisingly for very simple behaviours this also humans base their choices on Bayesian
inference. This was shown in a very simple experiment that you can do when you feel
bored: walk along the street with a friend and at a given time stop and look up at a 
point in the sky. After 1 hour you will get other 20 people doing that and so on.
This primitive communication language happens if, as we said before, all the necessary
conditions are met: a joint frame of attention for the pigeons looking for food on
the ground and for the humans looking at some point in the sky, the ability to
imitate actions (here we suppose to move to the same point in the space) and the
ability to remember what the other are doing. How this a-priory knowledge has
developed could depend on many factors, it's certainly due to the evolution
mechanism as well as on our personal experience. For us it's very hard to get
fooled by that trick more than one time as we may infer that when a group of
people gather in that situation it has no importance, but for the pigeon it is another
story. Pigeons will keep that behaviour because it is beneficial:
on average they will be able to find some food and therefore will keep
their a-priory estimate so that next time they will come back.
For the unlucky pigeon that, by a series of unfortunate events, will not get any food,
there will be a life of solitude and eventually an early retirement!
The conclusion here is that language cannot be based solely on actions because,
they are embodied in the environment where they are performed.
Thus to develop a more complex language, the organism must be provided with
an additional layer of sensory-motor loop designed or adapted for the purpose of communication.


\subsection{Language model and control}
A control model which includes language was used in \citet{Steels:1999:TalkingHeads}
and is summarised in Figure \ref{Fig:Language:Talk}
where two artificial agents play the role of the speaker and the hearer each
time they meet each other in a rule based language game.
The model adopted by Luc Steels includes similar features to Luhmann, the
intentionality is achieved by the goal to reduce uncertainty, the utterance module
also involves the information selection, and parsing the utterance is basically
the understanding process.

\begin{figure}[htbp]
\begin{center}
\includegraphics[width=0.8 \textwidth]{autopoiesis/languageModel}
\end{center}
\small{
\caption[Luc Steels language model]{
Both the speaker and the hearer have a layered architecture:
the language system takes care of the production and parsing of the utterance,
the conceptual system takes care of the understanding,
the sensory-motor system includes the modelling of the world and the intention
or goal of the agent.
\label{Fig:Language:Talk}}}
\end{figure}

In the talking head experiment \citep{Steels:1999:TalkingHeads}, every communication round
is composed of one speaker and one hearer.
The speaker and hearer share a whiteboard (called ``the context``)
full of geometric coloured shapes (triangles, squares, circles).
The speaker, uses image segmentation to choose a topic like ''the green triangle``
or ''the square in the top left corner``.
The speaker then, chooses a word from its dictionary to describe the topic,
then it emits a ''linguistic hint`` to the hearer.
Based on the linguistic hint, the hearer tries to guess what topic the speaker
has chosen, and he communicates his choice to the speaker by pointing to the object.
The hearer points by transmitting in which direction he is looking.
The game is considered successful if the topic guessed by the hearer is equal
to the topic chosen by the speaker.
The game fails if the guess was wrong or if the speaker or the hearer failed at
some earlier point in the game.
In case of failure, the speaker indicates the topic he had in mind, and both
agents try to ''synchronize`` their dictionaries to be more successful in future games.
The talking head experiment shows that a grounded language can indeed be
evolved and contains many properties seen in natural languages, such as polysemy and synonyms.

In my simulations what the agents are missing is the conceptual system layer
of Figure \ref{Fig:Language:Talk} which generated concepts and extract meanings
from the language system.
The world model of my agents is quite simple because it basically reduces a multidimensional
time series in a single integrated value which is the ICO weight.
The world model is of course a bottle neck for the conceptual system but that
does not imply that a simple conceptual layer cannot be developed.
For example part of the plan in my simulations was to introduce a ``telepathy'' feature
whereby the agents shares their own weight with each other.
When the agent receives the weight from one if his fellow, it can decide whether
to use it or not.
If the weight is beneficial, then is going to keep it otherwise is going to reject it.
This simple mechanism allow agents to share their representation of the world
and assigning a simple binary meaning: does it work or not?
With this sort of communication, one would also expect a faster convergence in the
self-organization property because agents does not have to experience everything
but can just try to apply somebody's else knowledge.
This feature unfortunately as explained in Figure \ref{Fig:Language:Telepahty}
was not implemented for the lack of time but would be
certainly improve the model complexity and possibly generate more complex collective
behaviour. 

\begin{figure}[htbp]
\begin{center}
\includegraphics[width=0.8 \textwidth]{Telepathy}
\end{center}
\small{
\caption[More advanced language model]{
In this situation there are three agents.
Agent 2 behaviour is to be attracted by agent 3 due to its current weight status $w12$.
Agent 1 is going to learn the food attraction behaviour and consequently
transmitting his weight $w11$ to Agent 2.
Agent 2 will receive the communication from Agent 1 and will decide to try
the new weight $w12$.
If the operation is successful it will keep the new weight $w12$ and overwrite
its previous one $w11$.
\label{Fig:Language:Telepahty}}}
\end{figure}


It is also equally important to study animal language and see what are the
main differences with the human language.
Due to the complexity of the cognitive abilities of humans, in this Thesis
I used models of language closer to animals rather then
humans, and so I am going to describe in the following sections some experimental
evidence of animal language.

\subsection{What is going on in an animal's head?}
What do the signals given by primates, and their responses, tell us about how monkeys think?
When we see an animal do something, it is tempting to assume that it's thinking as we
would if we behaved in the same way. But, as we explain in the context of alarm calls,
this need not to be so. Sometimes, however, by appropriate use of playback experiments,
we can get answers.
\subsection{Do signals convey information about the external world?}
Some signals convey information about the signaller. For example the black and yellow
stripes on a Cinnabar Moth caterpillar carry the message ''I am distasteful', a
fact about the signaller, not about the world external to the signaller.
In our model, agents show to the other, their state of satedness: it's the same concept.
But other signals do carry such information: for example, a bird alarm call carries the
message ``there is a predator close by'' or in my framework, the agent signals the
presence of a food resource in front of it by changing its colour. So what's the point?
It's about what, if anything, goes on in the mind of the receiver of the signal.
Is the receiver genetically programmed to react to the alarm (a pure reactive agent) or
is the receiver formulating a hypothesis of the external world (a non reactive agent)?
To be specific, when a Vervet Monkey hears a Leopard alarm, it climbs a tree.
Does it do so because it forms an image of a Leopard in its mind and behaves accordingly,
or because it follows the behavioural rule ``when you hear that call, climb a tree'' ?
We know that a Vervet will behave appropriately when it hears a Leopard alarm,
even when no Leopard is present.
But what is going on in its head? \citet{Seyfarth2000:AwarenessMonkey} attempted
to answer this question by habituating experiments.
Summarizing their conclusion ``Vervet Monkeys, therefore, appear to interpret
their calls as sounds that represent, or denote, objects and events in the external world''.
Current imaging studies are shedding more light into the mechanism of humanoid brains
 but there is still a lot of unknown processes.

\subsection{Do signallers intend to alter the behaviour of receivers?}
\label{TheoryOfMind}
\citet{Moller1998:FalseAlarmCalls} argues that subordinate birds gave false alarms to drive away more dominant
individuals and thereby gain access to food. It requires only that individual birds learn that
giving an alarm note increases their access to food: the calling bird does not
have to think ``if I give an alarm, other birds will think that there is a predator and fly away''.
An even more cautious interpretation is that the behaviour is not learnt at all:
it's innate in all situation in which calling has been selectively favoured in the past.
In my framework some agents cheat to decrease food competition, but to keep the model
simple this behaviour is not learned, only a percentage of the population is
cheating (see Section ``Animal theory: social signals''). In this example,
there is no need to assume that an animal ascribes thoughts and beliefs to others.
Humans certainly do. What of other primates?
A summary of Dennett's classification of ``intentionality'' \citep{Dennett1987:Intentional}:
\begin{itemize}
 \item \textit{Zero-order intentionality.} The signaller holds no beliefs or desires:
 a black and yellow caterpillar is a likely example.
 \item \textit{First-order intentionality.} The signaller holds beliefs,
but no beliefs about the beliefs of others.
A Great Tit giving an alarm does not believe that there is a predator (if the signal is honest),
or that the alarm will not increase its access to food (if the signal is a lie),
but in neither case need it have any beliefs about what other Great Tits are thinking.
 \item \textit{Second-order intentionality.}  The signaller ascribes thoughts and beliefs to the receiver.
\end{itemize}
The existence of zero-order and first-order intentionality in animals
should not be controversial. One problem is relevant for the second-order intentionality:
the influence on the signaller of the presence of potential signal hearers.
Vervets and others (including ground squirrels and chickens) do not call when alone.
This shows that animals may be aware of the presence of other individuals before giving an
alarm, but does not require that they ascribes thoughts to others. To summarise,
although animals are influenced, when signalling, by the presence and relatedness of
potential hearers, they do not seem to be influenced in their signals by the
knowledge that hearers might be supposed to possess (e.g. a monkey already giving the
alarm call is supposed to know that a leopard is present) Is there any evidence that
signallers ascribe beliefs to others? A theory  called ``Machiavellian Intelligence''
was formulated by \citep{Byrne1988}. The specific thesis is that
group-living primates have been selected to deceive other group members, and that this
requires that they develop a ``theory of mind'': they are able to ascribe beliefs to others.
There is a general agreement that primates do sometimes send signals which causes others
to behave as if they have been deceived. However, as said before about the hawk alarma call,
 this does not require that the signaller ascribes beliefs to others: it is sufficient that
 the signaller learns by experience that the false signal has the desired effect on
 the receiver's behaviour.
Human children, for example, develops the ability to ascribe beliefs when they are 4 years old
\citep{WimmerPerner1983:ChildMind}.
There is no doubt that some animal signals do potentially carry information
about the external world, and that receivers of such signals respond in a way
that would be appropriate if they had acquired that information. It is much harder
 to decide in particular cases whether the receiver in fact acquires
the information, or whether it merely responds appropriately.
According to the theory of animal signals \citep{AnimalSignals}, signals in my simulations can
 be regarded as:
\begin{enumerate}
\item the field $G_{sated}$ in eq. \ref{eq:gsated} is an index signal, expressing
a quality of the agent (its state of satedness) which cannot be faked.
\item the field $G_{food}$ in eq. \ref{eq:food} could be regarded as a costly
signal or a free signal
\end{enumerate}
I did two different simulations considering, $P_{e}(t)$ as the efficacy cost
needed to ensure that the information can be reliably perceived, $P_{s}(t)$ cost
 needed by the handicap principle \citep{Zahavi1975:MateSelection} to ensure honesty.
Efficacy cost is considered free in my simulation (see section \ref{SocialSystem:Broadcast}):
\begin{enumerate}
\item $G_{food}$ as an honest signal with $P_{e}(t)=0$ and $P_{s}(t)>0$ because competition for food will increase
\item $G_{food}$ as a dishonest signal with $P_{e}(t)=0$ and $P_{s}(t)=0$ because the agent that hasn't food, does not have any disadvantage.
\end{enumerate}
The model that I used is based on the zero order intentionality and one of the aim
for future research is to achieve the second order intentionality.



\section{Future work}
\label{Conclusion:FutureWork}
An important improvement for the model will be to use the symbolic language
module as described in section \ref{Conclusion:Language} with the double contingency
feature described in section \ref{Introduction:SocialOrderModel}.
With this approach one can have the power of a grounded language model and the
potential for social interaction to build an accurate model of Luhmann's society.
The $ICO/ISO$ learning controller could still be used to implement the action layer,
but other approaches will be required to implement the symbolic communication layer.
There are also some other extensions and considerations that can be included in future
models and are described in the following sections.


\subsection{Model based checking for property verification}
\label{Conclusion:ModelCheck}

The research on agent-based learning systems
currently relies on simulation results to infer the correctness of system properties.
These inferences are derived from averaging set of simulation results. 
However an alternative approached based on model checking can be used
to verify the properties of a learning system without the need of a simulation.
In a preliminary study with a fellow PhD student Ryan Kirwan from the computer
science department I have proved that it is indeed possible, with the correct
 abstraction model, to apply model checking to a dynamic learning agent and 
 prove some properties for a multi agent non learning system and a single
 learning agent.
The application of model checking to autonomous learning agents is novel
  and thus not straightforward.
  
\subsection{Economic models of learning agents}

Bayesian inference is a fundamental process behind human perception,
memory and cognitive judgement. While Bayesian inference has been investigated
 at the individual level, there are few studies regarding the implications of
 using Bayesian decision making in a multi agent social scenario.
\citet{Verschure98epistemol} argues that bayesian inference is an equivalent formulation to 
adaptive predictive control beacuse it is essentially a computational approach 
equivalent to the dynamical approach of neural based systems.
Thus it is possible to use Bayesian inference in a decision making task which requires
a selection between several actions i.e. an action policy.
An interesting economic social experiment can be setup where the goal of the
 artificial agent is to win a virtual English auction.
Every agent has a Bayesian predictive policy and/or an expectation about the
 others to decide the next move.
Therefore the simulated model takes into account the mix of subjective and
 social knowledge.
The social knowledge is based on the mutual expectation, a fundamental property
 of social systems. Luhmann hypothesised that high degree of behavioural dynamics
 can be achieved by using expectations as valuable knowledge for reducing
contingency about each others' behaviour and goals.
Therefore the author expects that a social based approach will generate
a realistic dynamic behaviour even in auction based games.
In the following sections the author describes a ABM system based on
 auction bidding that considers social expectations.

\subsection{Homogeneity in societies}
Luhmann did not pose any constraints on the homogeneity of social systems,
because his social model is essentially ``actor-free``.
This choice theoretical choice is quite good because it allows great flexibility
to build heterogeneous societies.
The only requirement is the need of a psychic system coupled to the communication
system.
For Luhmann a psychic system is a system able to generate thoughts, although there 
is a philosophical debate whether or not machines are able to 
think, dream or create, a psychic system can be implemented as a goal oriented behaviour
and a language module as proposed by Luc Steel.
It is useful therefore to distinguish between:
\begin{itemize}
 \item homogeneous societies: composed of identical entities, either artificial agents or humans
 \item inhomogeneous societies: composed of mixed entities like artificial agents and humans
\end{itemize}
There was a period of excitement in research after science fiction writers envisioned
 the integration of artificial intelligence entities in human societies.
Note that the very first "robots" in fiction, the neologism "robots" from Karel
Capek's R.U.R. \citep{Karel1920:RUR}, were actually Artificial Humans
 and not the clanking metal humanoids we now associate with that term.
A better and earlier term is Android from the greek andro- "human" + eides "form,
shape" meaning an "automaton resembling a human being". The term was first
mentioned by St. Albertus Magnus in 1270 and was popularized by the French writer
 Villiers in his 1886 novel L'Ève future.
There were and there still are efforts in producing androids which can be accepted
 by humans, avoiding the famous ``Uncanny Valley'' introduced by
Masahiro Mori as ``Bukimi no Tani Gensho''.
The latest androids are able to mimic the human aspect thanks to recent
advantages in material structures (artificial hairs, silicon skin)
The current problem is then to provide the androids with the intelligence to
interact socially and safely with humans.
Many researchers attempted the direct approach of designing social robots by
looking at the single human-robot interaction with an engineering top-down approach.
The design of social robots contains a variety of disciplines including:
mechatronic, science of materials, psychology, neuroscience, haptic interfaces,
voice recognition, speech synthesis, power systems etc.
It is certainly an interesting field, but it is mainly driven by technologist
and behavioural scientists,and thus is proceeding at a slow pace considering 
also the cost involved with building androids.

\subsection{Symmetry breaking in collective decision-making}

The self-organising property of social systems can be formulated
in terms of decision making because effectively one can imagine
that the agents has to make a collective choice about their division.
Collective decision making in social systems is often driven
by self-organising principles such as the choice of nest sites and food sources
by ant colonies and the aggregation of bees
\citep{Franks2003:StrategiesAnts,Dussutour2009:NoiseDecisionAnts,Meyer2008:noise-induced,
 Kornienko2009:ReembodimentOfHoneybee}.
Similar dynamics are present in bacteria colonies \citep{Reading2006:quorumbacteria} and even economic
markets \citep{Gerard2000:HitsFlopsDynamic}.

In my computational model the agent (individual) needs to decide whether to
obtain some food itself or steal the food from the others.
Because each agent has no initial preference or bias,
the agent needs to make a decision at the individual level
based on his memory (synaptic weight) and actual
sensory inputs (antennas).
Symmetry breaking means that the society will reach a majority
or unanimous decision.
In my case that implies that there will be a non equal distribution
of seekers and parasites.
It may not be obvious why the social system must always operate under symmetry breaking
even when there are 2 equally good sources.
This in fact happens in nature where for example many species
will converge on a single food source rather then equally distributing
between the 2 equally good food sources  \citep{Camazine2001:SelfOrgBiological}.

Symmetry breaking in self-organised decision making usually
arises from the interaction between positive and negative feedback loops.
The positive feedback in my model is the progressive weight increase
of the synapse which orientates the agent toward one behaviour,
for example food seeking.
The negative feedback in my model is the collision resulting from
a crowded group of agents going for the food.
The balance between these 2 systems has been shown to be a stable
and flexible enough decision system \citep{Dussutour2009:NoiseDecisionAnts,Meyer2008:noise-induced}.

The most common analysis of this coupled system is via differential equations,
but as stated before, the model is very complex to be described, especially
because the agents are active learner and thus their properties change
during time.

The alternative is to use a reduced statistical model which captures
only the relevant property of the symmetry breaking.
An early study in this field was done by \citep{Hamann2010:modelsimmetry} which described
the symmetry breaking in the honeybee behaviour and an emergent
density classification task with a simple stochastic differential equation.

\section{Information theory and control}
\label{Conclusion:Thoughts}

\subsection{On the perils of predictive learning}
Predictive learning is not the best solution for every situation. Why?
Because predictive learning is based on our subjective knowledge about the
world statistic in where we live. To give a clear explanation about when predictive
 learning fails we can think in probabilistic terms. Suppose we have a black box
that generates a stream of data, this can be the stock market, an auction on ebay
 or a football match. We don't know anything about the model behind the generation.
It could be in the worst case a markov process that generates events with maximum
 entropy. Nevertheless in the short run we only observe a causal relation between
 event A and event B, predictive learning that in the general formulation finds
the causal relationships between two events, A and B, and will infer that event B follows
 event A with probability 1! Predictive learning doesn't know the statistics behind
 the process generator because of its limited sampling capacities. If it was able
 to have an infinite sampling time (say the organism is immortal) it would experience
 all the possible pairings, and due to the ergodicity property of a markov process,
 it would infer that event A can be followed by event B with the same probability
 of being followed by event C.
This sampling problem can cause what we define in daily life as “hallucinations”:
a (conscious) perception in the absence of a stimulus.
In the absence of a stimulus, our predictive ability is reduced to zero and so everything
can be plausible and ``real''.
An extension of predictive learning models has been introduced by Schmidhuber \citep{Schuber2010:Novelty}:
\begin{quotation}
What's interesting? Many interesting things are unexpected, but not all unexpected
things are interesting or surprising. According to Schmidhuber's formal theory of
surprise \& novelty \& interestingness \& attention, curious agents are interested
in learnable but yet unknown regularities, and get bored by both predictable and
 inherently unpredictable things. His active reinforcement learners translate
mismatches between expectations and reality into curiosity rewards, or intrinsic
rewards for curious, creative or exploring agents which like to observe or create
truly surprising aspects of the world, in order to learn something new.
\end{quotation} \footnote{$http://www.idsia.ch/~juergen/interest.html$}
Schmidhuber rejects the original notion of the Boltzmann/Shannon surprise formulation from the 
early 1990s by posing two significant examples of uninteresting, unsurprising, boring data.
A vision-based agent that always stays in the dark will experience an extremely
 compressible, soon totally predictable and unsurprising history of unchanging
 visual inputs. In front of a screen full of white noise conveying a lot of
 information, "novelty" and "surprise", in the traditional sense of Boltzmann
 (1800s) and Shannon (1948), however, it will experience highly unpredictable
 and fundamentally incompressible data.
In both cases the data gets boring quickly as it does not allow for learning
 new things or for further compression progress.
Neither the arbitrary nor the fully predictable is truly novel or surprising/interesting - 
only data with still unknown but learnable statistical or
algorithmic regularities are. This is a very good argument and it would be
interesting to integrate the notion of maximisation of learning speed in
future social models.
For a more mathematical formalisation between entropy, learning and prediction, the Appendix
in Sections \ref{Appendix:PredictionAndLearning},\ref{Appendix:InfoForPrediction}.

\subsection{Prediction or evolution}
There are only two options to design artificial agents: either the designer 
can evolve reactive systems or adapt predictive systems to the environment.
Evolution has luckily selected organisms which infer the causal structure of
 their environment to make predictions of their future actions or equivalently of
 the future stimuli.
 Most researchers  will argue that I'm talking about different
 things, but a careful study of closed loop system can show that if the organism is
 able to predict what the next stimulus will be if he chooses an action, then he
 is also able to predict what the stimulus will be if he chooses not do anything.
Imagine a cat observing a little mouse running in front of it, the cat will estimate
 the trajectory taken by the mouse at a given time and will decide if is worth trying to pounce 
 on it or if the mouse is too fast and so waiting for a closer trajectory wastes less energy. 
 This is a well known probabilistic dilemma: when the organism needs to be reactive and when the organism 
 needs to learn?
 The Shannon information measure is a convex function, when expressed as $\sum_p p\cdot log_2(p)$ but 
 not if only expressed as $\sum_p log_2(p)$ which is infinite at $p=0$.
 So the best choice which maximizes the information is 
 at $p=0.5$ which gives a fifty fifty selection chance!
 This poses another question: if an organism wants to have
 a maximally predictive state of the environment, why do anything since a
 stationary state produces less possible entropy! If this assumption was correct
we would live in a stationary environment where organisms do very little as required
 by their survivor instinct. Well in the animal
world there are uncommon animals which live in very boring environments like
in the darkness of a deep ocean where their world is a flat surface with rare events
 happening without any clues. It turns out that the best organism is the one that is
very fast to react to changes, prediction has little sense in this game.
A star fish is the best choice, there is no need for huge brains with lot
of computational power because it will be mainly wasted. Coming back
 to our land we can see how higher complex organism have developed different
senses and very complicated brains to cope not with a complex environment but
with a causal environment. The common mistake is to think that a complex environment
 requires a complex organism. This is not true! Even a fairly simple organism
can generate a complex behaviour when placed in a complex environment.
Ashby proved that even when the inputs are connected to the outputs
with a simple proportional rule, the the variety will be transferred
from input to output unchanged. A superficial observer will say that this organism has 
a rather complicated behaviour! Moreover
 if we feedback the motor action into the input according to a function $f$,
coupling will generate an even more complex behaviour.  As already shown
by \citet{Kulvicius2009:analysisdifferential}, output entropy
was computed as the ration between the reflex output and the 
predictor output, allowing the author to separate the 2 contributions during learning.
This approach of separating the different output types was necessary to
avoid the afore mentioned problem of the variety transmission from inputs to outputs.
The ICO/ISO learning, in terms of information theory, is an internal
memory which integrates sensory information and to some extent compress the
sensory information by discovering the causal relation between the predictor and the reflex.
In this way then the total output entropy will not vary significantly and thus
cannot be used as a parameter of learning as well as complexity.



\subsection{Prediction and learning}
\label{Appendix:PredictionAndLearning}
The definition of predictive information: a quantity that measures how much our
observations of the past can tell us about the future. The predictive information
 describes the world we are experiencing and has a direct link to its complexity.
We as organisms collect sensory information in order to choose our actions
(including our verbal communication) but we are only interested in the data
 that tells us something about the state of the world at the time of our actions:
 non predictive information is useless to us. Surprisingly most of the information
 we collect over a long period of time is non predictive, so that isolating the
predictive information must extract from the sensory stream those features that
 are relevant for behaviour.
Definition of learning: finding a generalised model that explains or describes
 a set of observations. Why generalised?
Because we don't want to have an overfitted approximation of the data \citep{Vapnik1998:StatisticalLearningTheory}
 states that an animal can gain selective advantage not from its performance on the
 training data but only from its performance at generalisation.
Learning a model is also equivalent to encoding the data produced by it \citep{Rissanen1989:Complexity},
thus predicting and compressing are dual problems.
Complexity is an intuitive property ascribed to physical systems like turbulent flows,
ferromagnets materials etc...
Complexity is not random. Kolmogorov complexity states that a true random string
 cannot be compressed and hence requires a long description \citep{Kolmogorov1965:InfoDefinition}, yet
 the physical process that generates this string may have a very simple description.
Intuitively and from now on we refer to the complexity of the underlying process
 and not to the description length of the string generated from the process.
\citet{Bialek2001:Complexity} proved that predictive information $I_{pred}(T)$ provides
 a measure of complexity of the model underlying a time series. For small observation times:
\begin{equation}
I_{pred}(T,T')=H(T)+H(T')-H(T+T')\label{Ipredgeneral}
\end{equation}
 but in the limit of large observation times:
\begin{equation}
I_{pred}(T)=\lim_{T'\to\infty} I_{pred}(T,T')=H_{1}(T).
\end{equation}
$H(T)$ is the entropy computed on the signal $x(t)$ for $-T<t<0$ denoted in short
 hand by $x_{past}$, $H(T')$ is the entropy of the signal $x(t)$ that will be observed
 in the future $0<t<T'$ denoted in short hand by $x_{future}$. If the future and the
 past are statistically independent $P(x_{future}|x_{past})=P(x_{future})$
and viceversa $P(x_{past}|x_{future})=P(x_{past})$, then we cannot make any prediction:
the random guess is the best choice to predict the future. All predictions are
 probabilistic and so if the past tells us something about the future (and viceversa)
 we can use the conditional distribution of the future events on the past data:
$P(x_{future}|x_{past})$. Where the density $P(x_future|x_past)$ has smaller
entropy compared to the prior distribution $P(x_{past})$ means that there was
 a reduction in entropy and that the particular future event is more likely to happen.
The average of this predictive information is defined as:
\begin{eqnarray}
I_{pred}(T,T')=\sum_{past,future} P(future,past)\cdot \frac{P_{future,past}}{P(future) P(past)}\\
P(future,past)=P(future|past)\cdot P(past)
\end{eqnarray}
and can be rewritten as:
\begin{eqnarray}
 I_{pred}(T,T')=<log_2 [\frac{P(future|past)}{P(future)}]>\\
=-<log_2 P(future)> -<log_2 P(past)> -<[-log_2 P(future,past)]>\label{Ipredextense}
\end{eqnarray}
where $<...>$ denotes the average over the joint distribution of the past and the future.
Because the elements of the equation are all entropies we can rename the variables as:
\begin{itemize}
\item $-<log_2 P(future)>=H(T')$
\item $-<log_2 P(past)>=H(T)$
\item $-<log_2 P(future,past)>=H(T,T')$
\end{itemize}
thus using the new variables in \ref{Ipredextense} gives us the equation in \ref{Ipredgeneral}.
What the mutual information tell us?
$I_{pred}(T,T')$ is either the information that a data segment of duration $T$
provides about the future length $T'$ or the information that a data segment of
duration $T'$ provides about the immediate past of duration $T$.
Now the entropy of a time series is proportional to its duration asymptotically,
so that $\lim_{T\to\infty} H(T)/T=H_0$ thus entropy is an extensive quantity and
 predictability only depends on $H_1$ because:
\begin{equation}
I_{pred}(T,T')=H_0\cdot T + H_1(T) +H_0\cdot T' + H_1(T') -H_0\cdot (T+T')-H_1(T+T')
\end{equation}
and so $I_{pred}(T,T')=H_1(T)+H_1(T')$. Giving that:
\begin{eqnarray}
 H(T)=H_0 T + H_1(T)\\
\lim_{T\to\infty} H(T)/T=H_0\\
H_1(T)\geqslant 0\\
\lim_{T\to\infty} H_1(T)/T=0\\
\end{eqnarray}
and extending the future forwards toward infinity $T' \rightarrow \infty$ or the
 past towards minus infinity $T' \leftarrow -\infty$ the predictive information becomes:
\begin{eqnarray}
I_{pred}(T)=H_1(T), T' \rightarrow \infty\\
I_{pred}(T)=H_1(T), T \leftarrow -\infty\\
\end{eqnarray}
This equality states that there is symmetry between prediction and postdiction
 $I_{pred}(T,T')=I_{pred}(T',T)$ but also that the predictive information at
time $t=T$ gives us the same amount of information about the history of our
observation as well as the same amount for the future ones that will start from
 the present time.
\subsection{How much information is required for prediction?}
\label{Appendix:InfoForPrediction}
As we observe a time series for a long time $T$, we accumulate data which is measured
 by the entropy $H(T)$, and for $T$ that goes to infinity $H(T)\simeq H_0 T$.
Because the predictive information cannot grow linearly with time, only a small
 fraction of it is relevant for prediction:
\begin{equation}
\lim_{T\to\infty}\frac{PredictiveInformation}{TotalInformation}=\frac{I_{pred}(T)}{H(T)} \rightarrow 0
\end{equation}
although we collect data in proportion to our time $T$, a smaller and smaller fraction
 of this information is useful in the problem of prediction. Nevertheless this
property is true if the model that generates the time series has not changed its
 parameters, but what happens if the organism or somebody else ``a deus ex''
changes one of the parameters? The organism will experience a discontinuity
in the predictive information that indicates a novelty or better a surprise.
 
\subsection{Predictive information and model complexity}
In the regime of infinite observation time $T\rightarrow \infty$, $I_{pred}(T)$ can:
\begin{itemize}
 \item remain finite as $H_1=h_1$
 \item grow logarithmically $H_1=h_1 + k \cdot log(T)$
 \item grow as a fraction power law $H_1=h_1 + h_2 \cdot T^{\alpha}$
\end{itemize}
The first possibility \textbf{$\lim_{T\to\infty}=h_1$}, means that no matter how
 long we observe, we gain only a finite amount of information
The second possibility \textbf{$\lim_{T\to\infty}= k\cdot lnT$}, means that future
 observations depend on far distant past ones: the model that generates the time
 series has a number of finite parameters. The coefficient of the divergence $k$
counts the number of parameters of the model.
The third possibility \textbf{$\lim_{T\to\infty} \propto T^{\alpha}$}, means that
the underlying model has infinite parameters.
Estimation of the sub-linear component can be achieved using non-linear regression
 methods or using evolutionary fitting.

\subsection{Prediction and compression are related}
Suppose now that one of our agents is deprived of his output with the environment
 so that it can only observe a set of data: $x_{1},x_{2},...,x_{N}$. When we can
 say that the agent has learned? When the agent is able to predict the next
observation $x_{N+1}$ in case of 1 step prediction. The more an agent knows
the more accurate the prediction is about $x_{N+1}$ and the fewer bits are
required to describe the difference or error from the previous observations.
The average length of code required to describe the point $x_{N+1}$ given
the previous history:
\begin{equation}
 l(N)=-<log_2(P(x_{N+1}|x_1,x_2,...,x_N))>_{P(x_1,...,x_N,x_{N+1})} bits
\end{equation}
is the averaged conditional probability over the joint distribution of all the
 N+1 points. Remembering that the average code that describes a random sequence
 of $N$ samples is the entropy $H(N)$ of that sequence, we can write:
\begin{equation}
 l(N)=H(N+1)-H(N)\approx \frac{\partial H(N)}{\partial N}
\end{equation}
We learn more when we use a smaller description for the time series.
We can define a learning curve that measures the cost of encoding the next sample.
The ideal encoding's length can be known if the agent observes the stream of
 data for a infinite time:
$l_{ideal}=\lim_{T\rightarrow \infty} l(N)$,
thus the learning curve is the difference between the actual code length and the ideal length:
\begin{equation}
\varLambda \equiv l(N)-l_{ideal}=\frac{\partial I_{pred}(N)}{\partial N}
\end{equation}
The learning curve is the derivative of the predictive information and quantifies the
information learned so fare, if zero means that the optimal description code of the time series
is reached.


\subsection{Entropy reduction measure in learning agents}
\label{Conclusion:PredictiveBayes}
Predictive learning can be reduced to a probabilistic model and reformulated
in terms of Bayesian learning.
A simple model can be formulated using random discrete variables $Y,X$ and $W$.
$Y$ is a binary random variable that represents the distal signal ($Y={0,1}$) and
$X$ is a binary random variable that represents the reflex signal ($X={0,1}$)
where $Y=1$ means that the distal signal was active and $Y=0$ was not active.
In an open loop case when the agent cannot feedback his actions to the environment,
I can suppose that the reflex has the same probability of being present and
absent $P(X=0)=0.5$ and the same condition applies to the distal signal $P(Y=0)=0.5$.\\
Using a non symmetric distribution like $P(X=0)=0.58$ implies the presence of a
bias for the reflex to appear and indicates that the agent will do much work to compensate for that.

When the organism is regulating in a closed loop, a perfect regulator achieves
an entropy reduction of the reflex as mentioned previously:
\begin{equation}
P(X=0)=1 \rightarrow H(X)=0
\end{equation}
An imperfect regulator will be identified on the contrary by:
\begin{equation}
0 < P(X=0) < 1 \rightarrow H(X) > 0
\end{equation}
A similar measure of the effectiveness  of regulation could be the expectation of the variable $X$:
\begin{equation}
E(X)=\sum_{i=0}^{1} x_{i} \cdot p(x_i)
\end{equation}
Perfect regulation implies that $E(X)=0$, whereas imperfect regulation implies $E(X)\neq 0$ because 
the expectation $E(X)$ can be positive but also negative.

To be more clear, I can consider a better discretisation of the input space: $X=\{ -1,0,1\}$ 
mapping in this case a negative error, a zero error, and a positive error. 
If before learning $X$ has a uniform distribution like:
\begin{equation}
p(X_{before})=\{1/3,1/3,1/3 \}
\end{equation}
\begin{eqnarray}
E(X_{before})&=&-1\cdot \frac{1}{3}+0 \cdot \frac{1}{3}+1 \cdot \frac{1}{3}=0 \\
H(X_{before})&=&-3\cdot \frac{1}{3}log_2(1/3)=log_2(3)=1.5850 \; bits
\end{eqnarray}
But after learning or an equivalent successful perfect regulation where $p(X)=\{0,1,0\}$
\begin{eqnarray}
E(X_{after})&=&-0+1\cdot 1+0= 1 \\
H(X_{after})&=&log_2(1)=0 \; bits
\end{eqnarray}
Why is this so? Because entropy is a concave function of the distribution function p, 
whereas estimation is not able to distinguish between the two different steps of estimation.
In my predictive performance, I have used the entropy to estimate the predictive
performance because of the properties of entropy like non-negativity, concavity and
chain rules for mutual information.
However one does not have to exclude the expectation as a potential candidate for other
useful purposes.

Intuitively if X and Y are causally dependent the predictive controller can achieve optimal 
prediction, whereas if if X and Y are only statistically dependent it will achieve sub-optimal prediction.\\
When the agent experiences the world using his innate reflexes, it can observe that $P(X|Y)$ 
the probability of observing the reflex X is dependent on the probability of observing Y, 
in order words X and Y are not conditionally independent (if they were independent $P(X|Y)=P(X)*P(Y)$).\\
If the agent does not use the distal signal $w_1=0$, it will observe that $P(X=1|Y=1)=k1$ and 
that the $P(X=0|Y=0)=k2$ is possibly high. Learning is achieved when the agent selects the best 
action so that $P(X=1|Y=1)<k1$, an important fact of predictive learning is although it is 
desirable that $P(X=0|Y=0)>k2$ it is not possible to do that because of the impossibility of 
the correlator to evaluate the pairing of a missing distal event with a missing reflex event.\\

This problem of the correlation of missing events is a weakness in many learning algorithms,
 but there have been some new models, like the one developed by Ian Glascher who is testing 
a dual model based on the Wagner-Rescorla equation, which consider the positive rewarded 
outcome complementary to the negative rewarded outcome. In other words the model also considers
 what did not happen after the agent made a particular choice.
Predictive learning based on correlation therefore wants to choose actions so that the 
distribution of the events is:
\begin{itemize}
 \item before learning $P_{before}(X=1|Y=1)=0.6$,$P(X=0|Y=1)=0.4$
 \item after learning $P_{after}(X=0|Y=1)=0.8$, $P(X=1|Y=1)=0.2$
\end{itemize}
So the change in the distribution of $P_{k}(X=0|Y=1)$ is an index of the agent predictive power.
However I need also to consider how well the agent has learnt to avoid the reflex or equivalently to regulate itself.
I need to find an information measure which combines:
\begin{itemize}
 \item the predictive power of the agent
 \item the regulatory power of the agent
\end{itemize}
Since I want to use entropy for its attractive properties of concavity and non-negativity,
I can revisit the concept of conditioned entropy and mutual information and see if they suit our purposes.
Conditioned entropy:
\begin{eqnarray}
H(X|Y)&=&-\sum_x \sum_y p(x,y) log (p(x|y))\\
H(Y|X)&=&-\sum_x \sum_y p(x,y) log (p(y|x))\\
H(X|Y)&=& H(X,Y)-H(Y)\\
H(Y|X)&=& H(X,Y)-H(X)\\
\end{eqnarray}
Where the joint entropy $H(X,Y)$ is formulated as:
\begin{equation}
H(X,Y)=-\sum_x \sum_y p(x,y) log (p(x,y))\\
\end{equation}
and the mutual information as:
\begin{eqnarray}
I(X,X)&=&0\\
I(X,Y)&=&I(X,Y)\\
I(X,Y)&=&H(X)-H(X|Y)\\
I(X,Y)&=&H(X)+H(Y)-H(X,Y)\\
I(X,Y)&&\geqslant 0
\end{eqnarray}
In the next section I am going to evaluate which measure captures the learning performance of 
the agent considering typical probability distributions before and after learning.
The joint density can be represented as a matrix of 2 by 2 elements because in this case the 
variables X,Y are binary.
The table is constructed from the experimental data and must fulfil the properties
of probability distributions:
\begin{itemize}
 \item the integral of the joint distribution: $\sum_X \sum_Y p(x,y)=1$
 \item the integral of marginal distribution X: $\sum_X p(x)=1$
 \item the integral of marginal distribution Y: $\sum_X p(y)=1$
\end{itemize}
The Table \ref{table:beforelearning} shows a typical density distribution before learning,
considering the assumption of a uniform distribution for the reflex X and distal events Y,
and that the agent has not yet learned to avoid the undesired state $x=1$ using the
distal event $y=1$, thus $p(y=1,x=1)=0.4$. \\

\begin{table}[htbp]
\caption{
Entropy values before learning}
\label{table:beforelearning}
\begin{center}
\begin{tabular}{@{}c|ccc@{}}
\hline
  XY	   & $p(y=0)$ & $p(y=1)$ & $p(X)$\\
\hline
  $p(x=0)$ & $0.4$   & $0.1$   & 0.5\\
  $p(x=1)$ & $0.1$   & $0.4$   & 0.5\\
\hline
  $p(Y)$   & $0.5$   & $0.5$    & 1.0\\
\hline
\end{tabular}
\end{center}
\end{table}

From the table \ref{table:beforelearning}, I can compute the entropies in bits:
\begin{itemize}
 \item $H(X)=1 \; bit$
 \item $H(Y)=1 \; bit$
 \item $H(X,Y)=1.7219 \; bits$
 \item $H(X|Y)=1.7219-1=0.7219 \; bits$
 \item $I(X,Y)=1+1-1.7219=0.27810 \; bits$
\end{itemize}
The Table \ref{table:afterlearning} shows the density distribution after
learning was achieved: the agent swaps the rows in the column of $y=1$:

\begin{table}[htbp]
\caption{
Entropy values before learning}
\label{table:afterlearning}
\begin{center}
\begin{tabular}{@{}c|ccc@{}}
\hline
  XY	   & $p(y=0)$ & $p(y=1)$ & $p(X)$\\
\hline
  $p(x=0)$ & $0.4$    & $0.4$    & 0.8\\
  $p(x=1)$ & $0.1$    & $0.1$    & 0.2\\
\hline
  $p(Y)$   & $0.5$    & $0.5$    & 1.0\\
\hline
\end{tabular}
\end{center}
\end{table}

From the table \ref{table:afterlearning}, I can compute the entropy measures:
\begin{itemize}
 \item $H(X)=0.72193$
 \item $H(Y)=1$
 \item $H(X,Y)=1.7219$
 \item $H(X|Y)=1.7219-1=0.7219$
 \item $I(X,Y)=1+0.72193-1.7219=0.00003$
\end{itemize}
The mutual information after learning has been reduced to a very small number because 
$H(X,Y)$ is invariant to row or column permutations of the conditioned probability, 
but it is not the case for the marginal distributions $H(X)$ and $H(Y)$ that are 
changed because in this case $p(X=0)=0.8$ and $p(X=1)=0.2$. This means that the 
agent has learned to avoid the undesired state $X=1$.
An agent with perfect learning has a probability distribution as in Table \ref{table:perfect}:
\begin{table}[htbp]
\caption{
Entropy values before learning}
\label{table:perfect}
\begin{center}
\begin{tabular}{|c| c| c| c|}
\hline
  XY	   & $p(y=0)$ & $p(y=1)$ & $p(X)$\\
\hline
  $p(x=0)$ & $0.5$    & $0.5$    & 1.0\\
  $p(x=1)$ & $0.0$    & $0.0$    & 0.0\\
\hline
  $p(Y)$   & $0.5$    & $0.5$    & 1.0\\
\hline
\end{tabular}
\end{center}
\end{table}

When perfect learning is achieved the agent is always keeping the desired state 
$p(X=0)=1.0\rightarrow H(X)=0$, no matter what the distal event was $p(x=0,y=0)=p(x=0,y=1)=0.5$, 
hence $H(X,Y)=1$ and $H(Y)=0.5$.
However the entropy measure does not distinguish how to equivocate between an agent that
 learned ``a good thing'' from the one who learned ``a bad thing''.
For example in Table\ref{table:equivocation}, where the agent has swapped $p(x=0|y=0)$ 
with $p(x=1|y=0)$, it produces exactly the same values for the mutual information and 
the other measures but means that the agent has learnt to produce an action that, 
when the distal is present, evokes a reflex.\\

\begin{table}[htbp]
\caption{Equivocation table}
\label{table:equivocation}
\begin{center}
\begin{tabular}{|c| c| c| c|}
\hline
  XY	   & $p(y=0)$ & $p(y=1)$ & $p(X)$\\
\hline
  $p(x=0)$ & $0.1$    & $0.1$    & 0.2\\
  $p(x=1)$ & $0.4$    & $0.4$    & 0.8\\
\hline
  $p(Y)$   & $0.5$    & $0.5$    & 1.0\\
\hline
\end{tabular}
\end{center}
\end{table}

Therefore when considering learning, I need to have a look at the action policy: 
how the agent chooses an action to achieve the desired state. In our simulation the 
agent is properly wired so that it will compensate for the distal event but an 
improperly wired agent or an agent with a wrong action policy may reach non desired
 states while learning.
Therefore for a more general approach, I need to consider either the action 
policy or restrict the probability.
Intuitively, the combined system contains $H(X,Y)$ bits of information: we need H(X,Y) 
bits of information to reconstruct its exact state. If we learn the value of Y, 
we have gained $H(Y)$ bits of information, and the system has $H(X|Y)$ bits of 
uncertainty remaining. $H(X|Y) = 0$ if and only if the value of X is completely 
determined by the value of Y.
The Bayes theorem can be used to calculate the conditioned probabilities $p(y|x)$ 
from $p(x|y)$. This is more easy than computing $p(x|y)$ in our simulation because 
of the causal  temporal relation between $y$ and $x$ (y follows x).
\begin{eqnarray}
P(Y=0|X)=\frac{P(X|Y=0)\cdot P(Y=0)}{P(X|Y=0)+P(X|Y=1)}\\
P(Y=1|X)=\frac{P(X|Y=1)\cdot P(Y=1)}{P(X|Y=0)+P(X|Y=1)}\\
\end{eqnarray}
However this is not necessary if the estimation is made offline.

\section{Industrial applications}
\label{Conclusion:Industrial}
There are two industrial applications of social systems in the market right now.
One is the Kiva System \footnote{\url{http://www.kivasystems.com/}}, an automatic warehouse solution and the other is the Eporo system developed by Nissan.
The Kiva System is a group of robots which are placed in a warehouse and
can optimize the order fulfilment.
The Kiva robots are able to communicate with each other and with a sort
of control tower which assigns priorities to each robot.
The system is a very clear implementation of the advantages of using a
social approach to a traditional warehouse task.
The Eporo \footnote{\url{http://www.nissan-global.com/EN/NEWS/2009/_STORY/091001-01-e.html}} system is essentially a swarm behaviour implemented
in concept cars.
The main goal is to use the school fish behaviour to avoid accidents
in high density traffic. In this application there is no central controller
and thus it is more distributed but the communication is more on the action level.
The author is confident that in the future there will be more and more
social robotics implementation in the industry.
